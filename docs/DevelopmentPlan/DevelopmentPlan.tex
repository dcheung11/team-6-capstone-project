\documentclass{article}

\usepackage{booktabs}
\usepackage{tabularx}

\title{Development Plan\\\progname}

\author{\authname}

\date{}

%% Comments

\usepackage{color}

\newif\ifcomments\commentstrue %displays comments
%\newif\ifcomments\commentsfalse %so that comments do not display

\ifcomments
\newcommand{\authornote}[3]{\textcolor{#1}{[#3 ---#2]}}
\newcommand{\todo}[1]{\textcolor{red}{[TODO: #1]}}
\else
\newcommand{\authornote}[3]{}
\newcommand{\todo}[1]{}
\fi

\newcommand{\wss}[1]{\authornote{blue}{SS}{#1}} 
\newcommand{\plt}[1]{\authornote{magenta}{TPLT}{#1}} %For explanation of the template
\newcommand{\an}[1]{\authornote{cyan}{Author}{#1}}

%% Common Parts

\newcommand{\progname}{ProgName} % PUT YOUR PROGRAM NAME HERE
\newcommand{\authname}{Team \#, Team Name
\\ Student 1 name
\\ Student 2 name
\\ Student 3 name
\\ Student 4 name} % AUTHOR NAMES                  

\usepackage{hyperref}
    \hypersetup{colorlinks=true, linkcolor=blue, citecolor=blue, filecolor=blue,
                urlcolor=blue, unicode=false}
    \urlstyle{same}
                                


\begin{document}

\maketitle

\begin{table}[hp]
\caption{Revision History} \label{TblRevisionHistory}
\begin{tabularx}{\textwidth}{llX}
\toprule
\textbf{Date} & \textbf{Developer(s)} & \textbf{Change}\\
\midrule
Date1 & Name(s) & Description of changes\\
Date2 & Name(s) & Description of changes\\
... & ... & ...\\
\bottomrule
\end{tabularx}
\end{table}

\newpage{}

\wss{Put your introductory blurb here.  Often the blurb is a brief roadmap of
what is contained in the report.}

\wss{Additional information on the development plan can be found in the
\href{https://gitlab.cas.mcmaster.ca/courses/capstone/-/blob/main/Lectures/L02b_POCAndDevPlan/POCAndDevPlan.pdf?ref_type=heads}
{lecture slides}.}

\section{Confidential Information?}

\wss{State whether your project has confidential information from industry, or
not.  If there is confidential information, point to the agreement you have in
place.}

\wss{For most teams this section will just state that there is no confidential
information to protect.}
\section{IP to Protect}

\wss{State whether there is IP to protect.  If there is, point to the agreement.
All students who are working on a project that requires an IP agreement are also
required to sign the ``Intellectual Property Guide Acknowledgement.''}

\section{Copyright License}

\wss{What copyright license is your team adopting.  Point to the license in your
repo.}

\section{Team Meeting Plan}

\wss{How often will you meet? where?}

\wss{If the meeting is a physical location (not virtual), out of an abundance of
caution for safety reasons you shouldn't put the location online}

\wss{How often will you meet with your industry advisor?  when?  where?}

\wss{Will meetings be virtual?  At least some meetings should likely be
in-person.}

\wss{How will the meetings be structured?  There should be a chair for all meetings.  There should be an agenda for all meetings.}

\section{Team Communication Plan}

\wss{Issues on GitHub should be part of your communication plan.}

\section{Team Member Roles}

\wss{You should identify the types of roles you anticipate, like notetaker,
leader, meeting chair, reviewer.  Assigning specific people to those roles is
not necessary at this stage.  In a student team the role of the individuals will
likely change throughout the year.}

\section{Workflow Plan}

\begin{itemize}
	\item How will you be using git, including branches, pull request, etc.?
	\item How will you be managing issues, including template issues, issue
	classification, etc.?
  \item Use of CI/CD
\end{itemize}

\section{Project Decomposition and Scheduling}

\begin{itemize}
  \item How will you be using GitHub projects?
  \item Include a link to your GitHub project
\end{itemize}

\wss{How will the project be scheduled?  This is the big picture schedule, not
details. You will need to reproduce information that is in the course outline
for deadlines.}

\section{Proof of Concept Demonstration Plan}

What is the main risk, or risks, for the success of your project?  What will you
demonstrate during your proof of concept demonstration to convince yourself that
you will be able to overcome this risk?

\section{Expected Technology}

\wss{What programming language or languages do you expect to use?  What external
libraries?  What frameworks?  What technologies.  Are there major components of
the implementation that you expect you will implement, despite the existence of
libraries that provide the required functionality.  For projects with machine
learning, will you use pre-trained models, or be training your own model?  }

\wss{The implementation decisions can, and likely will, change over the course
of the project.  The initial documentation should be written in an abstract way;
it should be agnostic of the implementation choices, unless the implementation
choices are project constraints.  However, recording our initial thoughts on
implementation helps understand the challenge level and feasibility of a
project.  It may also help with early identification of areas where project
members will need to augment their training.}

Topics to discuss include the following:

\begin{itemize}
\item Specific programming language
\item Specific libraries
\item Pre-trained models
\item Specific linter tool (if appropriate)
\item Specific unit testing framework
\item Investigation of code coverage measuring tools
\item Specific plans for Continuous Integration (CI), or an explanation that CI
  is not being done
\item Specific performance measuring tools (like Valgrind), if
  appropriate
\item Tools you will likely be using?
\end{itemize}

\wss{git, GitHub and GitHub projects should be part of your technology.}

\section{Coding Standard}

\wss{What coding standard will you adopt?}

\newpage{}

\section*{Appendix --- Reflection}

\wss{Not required for CAS 741}

The purpose of reflection questions is to give you a chance to assess your own
learning and that of your group as a whole, and to find ways to improve in the
future. Reflection is an important part of the learning process.  Reflection is
also an essential component of a successful software development process.  

Reflections are most interesting and useful when they're honest, even if the
stories they tell are imperfect. You will be marked based on your depth of
thought and analysis, and not based on the content of the reflections
themselves. Thus, for full marks we encourage you to answer openly and honestly
and to avoid simply writing ``what you think the evaluator wants to hear.''

Please answer the following questions.  Some questions can be answered on the
team level, but where appropriate, each team member should write their own
response:


\begin{enumerate}
    \item Why is it important to create a development plan prior to starting the
    project?
    \item In your opinion, what are the advantages and disadvantages of using
    CI/CD?
    \item What disagreements did your group have in this deliverable, if any,
    and how did you resolve them?
\end{enumerate}

\newpage{}

\section*{Appendix --- Team Charter}


\subsection*{External Goals}

Our team's external goals include:

\begin{itemize}
    \item Achieve a grade of 11 or greater in the capstone course.
    \item Produce a project that team members can discuss in future job interviews.
    \item Produce a project that team members can take pride in as a representation of 4 years of Software Engineering courses.
    \item Gain experience in project management and working as a cohesive team.
    \item Build strong, professional connections within the team, with faculty/supervisors.

\end{itemize}

\subsection*{Attendance}


\subsubsection*{Expectations}

Our team's expectations regarding attendance are as follows:
\begin{itemize}
    \item Team members must attend all scheduled meetings, both virtual and in-person, unless a valid reason is provided in advance.
    \item Team members are expected to arrive on time and stay for the entire duration of the meeting. Repeated lateness or early departures without prior notice or valid reasons are unacceptable.
    \item Team member that are missing a meeting must notify the team at least 24 hours in advance, or as early as possible, except in the case of emergencies.
    \item Team members should come prepared, having completed any assigned tasks and actively participating in the productivity of the meeting.
\end{itemize}

\subsubsection*{Acceptable Excuse}

Acceptable excuses for missing a meeting or arriving late:
\begin{itemize}
    \item Health-related issues.
    \item Family emergencies or urgent personal matters.
    \item Conflicts with other courses or prior academic or professional commitments.
    \item Unavoidable technical difficulties.
\end{itemize}
Unacceptable excuses:
\begin{itemize}
    \item Forgetting the meeting time or oversleeping.
    \item Non-urgent personal commitments (e.g., social events, non-critical errands).
    \item Lack of preparation for the meeting.
    \item Laziness.
\end{itemize}

\subsubsection*{In Case of Emergency}

Our teams process for emergencies:
\begin{itemize}
    \item The team member must inform the team as soon as possible through the groupchat explaining their situation.
    \item If the team member cannot attend the meeting, they should provide a brief update on their progress or any relevant information so the team can continue without them (as soon as safely and conveniently possible).
    \item In the case that a deadline cannot be met, the team member must alert the team and the team will confer on the urgency of the task and reassign tasks accordingly.
    \item For major deliverables, the team may confer to try to accommodate a reasonable extension. Team members are expected to be willing to collaborate and contribute extra work to help cover for the lost time from the emergency and put the team first.
    
\end{itemize}
In case of prolonged emergencies, the team will review the situation and may reassign responsibilities to ensure the project remains on track.


\subsection*{Accountability and Teamwork}

\subsubsection*{Quality} 

Our teams process and expectations for quality:
\begin{itemize}
    \item Each team member is expected to come fully prepared for meetings, having completed any assigned tasks and ideation processes. 
    \item Deliverables and tasks should be done to the best of the team member's abilities.
    \item At least one other team member will have reviewed each task. If a task is below the teams accepted quality, the reviewer is expected to respectfully bring it to the attention of the team and suggest improvements.
    \item Team members are expected to provide constructive criticism to other team members, and team members are expected to handle the feedback professionally. 
\end{itemize}

\subsubsection*{Attitude}

Our teams process and expectations for attitude:
\begin{itemize}
    \item Each team member is expected to act professionally and cooperatively.
    \item Team members should be expected to be willing to help out on other member's tasks, providing feedback and support (within reason).
    \item Team members should follow the code of conduct and conflict resolution plans from Harvard University:  https://hr.harvard.edu/staff-personnel-manual/employee-conduct
\end{itemize}


\subsubsection*{Stay on Track}

To keep the team on track, we will:
\begin{itemize}
    \item Have weekly progress check-ins to ensure tasks are being completed as expected.
    \item Set specific target metrics, such as attendance, commits, and task completion. 
\end{itemize}

If any team member misses a meeting without reason, they have to bring coffee at the next meeting.

\subsubsection*{Team Building}

To build team cohesion, we will organize at least 2 team-building activities over the course of the Capstone course and begin meetings with a personal check-in.

\subsubsection*{Decision Making} 

Our team will aim for decision-making by consensus, where every member's opinion is considered. In situations where consensus cannot be reached, we will vote, with the majority decision winning. For significant disputes, we will seek input from the project TA or instructor to guide the final resolution.
\end{document}