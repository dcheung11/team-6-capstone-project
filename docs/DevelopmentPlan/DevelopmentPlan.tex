\documentclass{article}

\usepackage{booktabs}
\usepackage{tabularx}
\usepackage{parskip}

\title{Development Plan\\\progname}

\author{\authname}

\date{}

%% Comments

\usepackage{color}

\newif\ifcomments\commentstrue %displays comments
%\newif\ifcomments\commentsfalse %so that comments do not display

\ifcomments
\newcommand{\authornote}[3]{\textcolor{#1}{[#3 ---#2]}}
\newcommand{\todo}[1]{\textcolor{red}{[TODO: #1]}}
\else
\newcommand{\authornote}[3]{}
\newcommand{\todo}[1]{}
\fi

\newcommand{\wss}[1]{\authornote{blue}{SS}{#1}} 
\newcommand{\plt}[1]{\authornote{magenta}{TPLT}{#1}} %For explanation of the template
\newcommand{\an}[1]{\authornote{cyan}{Author}{#1}}

%% Common Parts

\newcommand{\progname}{ProgName} % PUT YOUR PROGRAM NAME HERE
\newcommand{\authname}{Team \#, Team Name
\\ Student 1 name
\\ Student 2 name
\\ Student 3 name
\\ Student 4 name} % AUTHOR NAMES                  

\usepackage{hyperref}
    \hypersetup{colorlinks=true, linkcolor=blue, citecolor=blue, filecolor=blue,
                urlcolor=blue, unicode=false}
    \urlstyle{same}
                                


\begin{document}

\maketitle

\begin{table}[hp]
\caption{Revision History} \label{TblRevisionHistory}
\begin{tabularx}{\textwidth}{llX}
\toprule
\textbf{Date} & \textbf{Developer(s)} & \textbf{Change}\\
\midrule
Date1 & Name(s) & Description of changes\\
Date2 & Name(s) & Description of changes\\
... & ... & ...\\
\bottomrule
\end{tabularx}
\end{table}

\newpage{}

\wss{Put your introductory blurb here.  Often the blurb is a brief roadmap of
what is contained in the report.}

\wss{Additional information on the development plan can be found in the
\href{https://gitlab.cas.mcmaster.ca/courses/capstone/-/blob/main/Lectures/L02b_POCAndDevPlan/POCAndDevPlan.pdf?ref_type=heads}
{lecture slides}.}

\section{Confidential Information?}

\wss{State whether your project has confidential information from industry, or
not.  If there is confidential information, point to the agreement you have in
place.}

\wss{For most teams this section will just state that there is no confidential
information to protect.}
\section{IP to Protect}

\wss{State whether there is IP to protect.  If there is, point to the agreement.
All students who are working on a project that requires an IP agreement are also
required to sign the ``Intellectual Property Guide Acknowledgement.''}

\section{Copyright License}

\wss{What copyright license is your team adopting.  Point to the license in your
repo.}

\section{Team Meeting Plan}

The team will meet twice a week, with a strong emphasis on maintaining at least one in-person meeting whenever possible. Physical meeting locations and continuous discussion outside of meeting times will be communicated internally over text messaging or the team’s designated communication platform (Discord)\par

Meetings will be organized as follows:

\begin{itemize}
  \item Progress Meeting: Focused on updates, task delegation, and alignment on overall project goals (minimum 30 minutes).
  \item Working Session: Dedicated to collaborative tasks such as coding, debugging, and addressing technical challenges (minimum 1 hour).
\end{itemize}

\begin{flushleft}
  We will meet with our industry advisor biweekly to discuss progress and challenges. In-person meetings with our advisor will always be prioritized. During weeks when a deliverable is due, this meeting shall take place at least 2 days before the deadline to allow for necessary feedback and revisions. 
  In-person meetings will occur at Hatch when rooms are available for booking.
  Virtual meetings will be held over MS Teams. All meetings will be organized email with the industry advisor with at least 24 hours of notice.\par

  Meeting Structure:
\end{flushleft}

\begin{itemize}
  \item Meeting Chair: A rotating team member will serve as the chair for each meeting, responsible for maintaining structure and guiding discussions.
  \item Agenda: A predefined agenda will be prepared by the chair and shared with all team members at least 24 hours in advance. This will allow everyone to review and prepare for the meeting topics.
  \item Minutes: A designated notetaker (rotating role) will document key decisions, progress, and action items.
\end{itemize}

In case of emergencies or immediate concerns, impromptu meetings can be scheduled at short notice using text messaging or the team’s designated communication platform (Discord). These meetings will be used to address urgent issues that cannot wait until the next scheduled session.
\section{Team Communication Plan}

Primary Communication Platforms:\par

\begin{flushleft}
  Instant Messaging: The team will use an instant messaging group chat for casual discussions or time-sensitive communication.\par

  Discord: The team will use Discord as the primary tool for daily communication, task updates, and file sharing. Different channels will be created for specific topics, such as:
\end{flushleft}

\begin{itemize}
    \item \#general: For overall team communication.
    \item \#progress-updates: For regular project updates and task completion.
    \item \#coding-issues: For discussing code-related problems or bugs.
    \item \#advisor-meetings: For coordinating and preparing for advisor meetings.
\end{itemize}

\begin{flushleft}
  GitHub: GitHub will be the main platform for tracking code development and issues. Each team member will be responsible for regularly pushing code and using the issues board to log any bugs or features that need to be addressed.\par

  Issue Tracking: All team members will log bugs, feature requests, and tasks as issues in GitHub. Each issue will include:
\end{flushleft}

\begin{itemize}
    \item A clear description of the problem or task.
    \item Labels to indicate priority (e.g., high, medium, low) and type (bug, documentation, etc.).
    \item A due date or milestone when applicable.
    \item Assigned member(s) responsible for it's completion when applicable.
\end{itemize}

\begin{flushleft}
  Meeting Summaries:
  After every meeting (team or with advisors), a summary of key points and decisions will be posted to Discord by the notetaker in \#advisor-meetings. These summaries will also be stored in a shared Google Drive folder for easy reference.\par

  Code Reviews:
  Pull Requests (PRs): Before any code is merged into the main branch on GitHub, team members will submit PRs for organization and traceability. Each PR must be reviewed and approved by 2 other members.\par

  Decision Making:
  For decisions on the project (technical or otherwise), team members will communicate through Discord or instant messaging. If consensus isn’t reached after discussion, a vote will be held, and majority decisions will move forward.\par
\end{flushleft}

\section{Team Member Roles}

To ensure that responsibilities are clear and all aspects of the project are efficiently managed, the following roles will be defined for each stage of the project. These roles will rotate as needed to provide equal opportunities for team members to lead and contribute across different areas:

\subsection{Project Manager} 
  \begin{itemize}
    \item Oversee overall project progress and ensure deadlines are met.
    \item Coordinate tasks and delegate responsibilities to team members.
    \item Monitor team morale and ensure smooth collaboration.
  \end{itemize}

\subsection{Team Liaison}
  \begin{itemize}
    \item Act as the primary point of contact with the faculty.
    \item Distribute and document all comminucation between the faculty to the rest of the members via text messaging or in an appropriate channel on Discord.
    \item Respond to emails from faculty in a timely manner and ensure all statements made to the faculty are approved by the rest of the group.
  \end{itemize}


\subsection{Meeting Chair} 
  \begin{itemize}
    \item Lead and structure team meetings.
    \item Prepare and distribute the meeting agenda at least 24 hours before the meeting.
    \item Ensure meetings stay on topic and follow the agenda.
    \item Facilitate discussions and ensure all team members are heard.
    \item Encourage participation from all members. Attempt to ask each member an engaging question at least once throughout the meeting.
  \end{itemize}

\subsection{Notetaker}
  \begin{itemize}
    \item Take detailed notes during team and advisor meetings.
    \item Document key decisions, action items, and deadlines.
    \item Distribute meeting minutes promptly after each meeting in the appropriate Discord channel (within 24 hours).
  \end{itemize}

  \subsection{Code Reviewer}
  \begin{itemize}
    \item Review code submitted by team members via GitHub pull requests to ensure quality and consistency.
    \item Provide constructive feedback and suggest improvements to ensure best practices are followed.
    \item Ensure the code adheres to the team's coding standards and works as expected.
  \end{itemize}

\subsection{Documentation Lead}
  \begin{itemize}
    \item Maintain and update project documentation, including the development plan, technical documentation, and user guides.
    \item Ensure that all team members contribute to the documentation as needed.
    \item Lead final reports and project summaries required for deliverables.
    \item Stay informed and up to date to act as an informant for other group members.
  \end{itemize}

\section{Workflow Plan}

To ensure efficient collaboration and seamless project development, the following workflow processes will be adopted:

\subsection{Git Workflow}
The team will follow the Git Feature Branch Workflow, but instead of "feature" branches, we will refer to them as "task" branches to better reflect the variety of work being done (e.g., adding documentation, fixing bugs, or developing new features).

\begin{itemize}
    \item Main Branch: The main branch will always contain production-ready code. Only tested and approved code will be merged into this branch.
    \item Development Branch: All new tasks, whether for coding, documentation, or bug fixes, will be developed in task branches created from the development branch.
    \item Task Branches: Each new task (feature, bug fix, or documentation section) will have its own dedicated branch named descriptively (e.g., webpage-login-feature, webpage-signout-fix). This allows team members to work independently without affecting the main codebase.
    \item Pull Requests (PRs): Once a task is complete, the developer will submit a pull request (PR) to merge the code into the development branch. PRs must be reviewed by at least two rotating team members before merging to ensure code quality and consistency.
\end{itemize}

\subsection{Issue Management}
\textbf{GitHub Issues} will be used to track tasks, bugs, and feature requests throughout the project. Each issue will:
\begin{itemize}
    \item Be clearly described, outlining the task, bug, or enhancement in detail.
    \item Include relevant labels to indicate priority (e.g., high, medium, low) and type (bug, task, improvement).
    \item Have a designated assignee to ensure accountability.
    \item Include milestones and deadlines where applicable.
\end{itemize}\par

\textbf{Issue Templates} will be created to standardize how issues are reported:
\begin{itemize}
    \item Bug Report Template: A form to report bugs, with fields for reproduction steps, expected behavior, and actual behavior.
    \item Task/Request Template: A form to describe tasks, such as documentation sections or new features, detailing the goal and steps for completion.
\end{itemize}

\subsection{Continuous Integration (CI)}
The team will implementContinuous Integration (CI) to ensure that the code remains stable as new changes are introduced.
\begin{itemize}
    \item A basic CI pipeline (using a tool like \textbf{GitHub Actions}) will be configured to automatically run tests whenever a new pull request is opened. This will ensure that the new code doesn’t break existing functionality or introduce bugs.
    \item The CI pipeline will run simple unit tests and style checks, ensuring the code is clean and functional.
    \item Only code that passes all CI checks will be eligible for merging into the development branch.
\end{itemize}

\subsection{Code Review Process}
Every pull request will require at least two reviews from rotating team members before being merged.
\begin{itemize}
    \item Rotating Reviewers: Team members will take turns reviewing pull requests to ensure that everyone contributes to maintaining code quality.
    \item Review Feedback: Reviewers will provide constructive feedback through GitHub, and any requested changes must be implemented before approval.
\end{itemize}

\subsection{GitHub Projects \& Kanban Board}
The team will use \textbf{GitHub Projects} with a kanban board to visually track progress. GitHub Projects allows automatic creation of tasks on the Kanban board from GitHub Issues, ensuring that both issue tracking and task tracking are seamlessly integrated. The Kanban board will include the following columns:
\begin{itemize}
    \item To Do: Automatically populated with new issues created in GitHub.
    \item In Progress: Tasks/issues that are actively being worked on will be moved to this column.
    \item In Review: Pull requests awaiting review will be placed here.
    \item Completed: Once a task is fully completed and merged, it will be moved to the Completed column.
\end{itemize}

\subsection{Task Assignment}
Tasks will be assigned via GitHub Issues and reflected on the kanban board. Team members can self-assign tasks or be assigned by the project manager. Larger tasks will be broken down into smaller sub-tasks to ensure steady progress. This process ensures that task assignment, status, and completion are always visible, helping the team stay on track and maintain accountability.

\section{Project Decomposition and Scheduling}

\begin{itemize}
  \item How will you be using GitHub projects?
  \item Include a link to your GitHub project
\end{itemize}

\wss{How will the project be scheduled?  This is the big picture schedule, not
details. You will need to reproduce information that is in the course outline
for deadlines.}

\section{Proof of Concept Demonstration Plan}

\subsection{Project Risks}

\begin{itemize}
  \item \textbf{Login, Registration, and User Roles:} Establishing a secure login and registration system while ensuring proper role assignments for users, captains, commissioner levels can present challenges, particularly in preventing unauthorized access and special permissions.
  \item \textbf{Scheduling:} Managing game schedules across multiple divisions and handling rescheduling due to conflicts can introduce complexities in both logic and user interface design. It may be tricky to implement schedule generation, game rescheduling, team preferences, etc. Some research on schedule generation is required.
  \item \textbf{Deployment and Scalability:} Deploying the application in a secure and stable manner also entails ensuring the platform is reliable under real-world conditions, especially as traffic increases during the season, and that updates or patches don’t introduce bugs or downtime. Approximately 1000 unique participants across 30-40 teams are using the current web page, and scalability must be acknowledged in design and implementation.
  \item \textbf{Roster Management:} Difficulty in implementing the "Request To Join Team" functionality for players and sending email notifications to the captains, as well as general roster management available to the captain.
\end{itemize}
\subsection{Proof of Concept Demonstration}

To address these risks, the PoC demonstration will focus on the following key areas to mitigate these concerns:

\begin{itemize}
	\item \textbf{Login, Registration, and User Roles:} \\ Goal: A working registration page where users can create accounts, log in, and see different views based on their roles. A demo admin will be able to access restricted settings while regular users can only view their team pages. 
  \item \textbf{Scheduling:} \\ Goal: A minimal working scheduling system that auto-generates a schedule for all divisions with an interface allowing captains to request a rescheduled game and set the preferred time slots.
  \item \textbf{Deployment and Scalability:} \\ Goal: A live deployment of the minimal POC on Heroku or similar hosting platforms, demonstrating real-time accessibility and showcasing a the ability for simultaneous user interactions with the platform.
  \item \textbf{Roster Management:} \\ Goal: Demonstrate a simple roster management interface where team captains can invite (and manage) players, with a live update of the team's roster, and players can request to join teams, notifying the captain. 
\end{itemize}

\section{Expected Technology}

Since the McMaster Softball League Platform will be a full-stack web development project, it is best to use technologies that are popular in the industry and ones that we are already familiar with. We choose to use modern frameworks so that this project can be easily picked up by someone else in the future if updates or modifications are needed. Our project needs to be highly maintainable and able to scale in size for long-term use.

\subsection*{Frontend}

\subsubsection*{React.js}

A JavaScript library for building interactive user interfaces. It's advantageous when we need dynamic parts of the platform, such as user interactions. This library will make it easier to control team management, scorekeeping, and any other functional interaction required to make the platform meet expectations. \\

\noindent Another important concept in React is components. We can create components to separate logic and these components can be reused constantly for different purposes. For example, we can design a simple button component and reuse this component at multiple endpoints for the site, saving time and increasing work efficiency.
\subsubsection*{Tailwind CSS}

A CSS framework that provides flexibility in styling the front-end components by using a utility-first concept. A utility-first concept is when you use pre-defined classes (utilities) directly in your HTML. Hence, we apply numerous pre-defined classes to achieve what we want. This framework saves time by eliminating the need to create new CSS classes for different designs, ensuring that everyone uses the same existing classes to maintain consistency across the website.

\subsection*{Backend}

\subsubsection*{Node.js with Express.js}

A scalable JavaScript runtime for building the server-side logic. Express.js will be used to manage API requests, handle authentication, and interact with the database. \\

\noindent Node.js is a runtime environment, which allows us to run JavaScript on our server. This tool is incredibly useful since it's practically mandatory for us to run out of backend logic for our application outside of the browser. \\

\noindent Express.js is a framework of Node.js, offering tools to simplify the web application development experience. This framework allows for an easier way to handle communication between our browser and server, such as GET requests and POST requests.

\subsubsection*{MongoDB}

A NoSQL database is ideal for a sports league platform like ours. MongoDB is known for being highly scalable and capable of handling large amounts of unstructured Data. Its flexibility is unmatched. \\

\noindent SQL stores rows in tables. In NoSQL, documents store information and can have different structures. Documents can be stored in the same collection, similar to a table. For example, one document might have a "coach" field, while another might not. However, they can both exist in the same collection.

\subsection*{Version Control}

Version control such as git, GitHub, and GitHub projects is mandatory since we don't want changes pushed to production without proper review. We also want our main branch to succeed, so new features must be developed on a separate branch to avoid unwanted errors and results. This practice will allow our developers to test freely without fear of messing up the work that everyone has done.

\subsection*{Authentication}

JWT (JSON Web Tokens) is the standard for secure user authentication and we can use this to vary access control depending on the user role. For our platform, permissions will be different for players, captains, and site admins.

\subsection*{DevOps \& Deployment}

We can use GitHub Actions to set up Continuous Integration (CI) and Continuous Deployment (CD). Although we are not as familiar with it, Github provides loads of documentation that we can read to learn about its features.\\

\noindent Since the old website is already being hosted in some domain, we can just reuse the same domain for our production. We don't need other deployments since we want to minimize funding.

\subsection*{Testing}

Playwright is the most popular and modern framework for end-to-end web testing. It provides fast execution times and provides many configuration options, such as running test cases in parallel. The platform also works well across different devices and user scenarios. The syntax is very intuitive and the documentation for this framework is very straightforward. Unlike Selenium, we won't need to install any web drivers and struggle with difficult setups.

\section{Coding Standard}

Similar to any engineering project, there are practices and rules that our web development project must adhere to. This list will contain the rules that we want to follow to make our project maintainable and scalable long-term for the present and future.

\subsection*{Consistent Code}

Everyone must write code that is easy to read and understand for others. Examples of this could be concise and descriptive variable names, function names, and consistent formatting of code, such as spaces on new lines instead of jamming them together on one. Code should be broken down smaller with an identifiable purpose so that it can be easily tested. We also do not want to duplicate code to avoid wasting time in the future.

\subsection*{SOLID}

Our code will follow the SOLID principles, which we learned in our previous software engineering classes. These principles are standard practices, ensuring that any code we write will be easily readable by others for interpretation or future usage.

\subsubsection*{Single Responsibility Principle}

A class should only have one responsibility. Responsibilities should obviously be separate to avoid confusion and code duplication. This principle also makes it easier to locate specific functionalities since we're purposely separating them.

\subsubsection*{Open/close Principle}

Open for extension, closed for modification. This principle means that any time we want to add a new functionality, there should be no need to change or modify the original code. This principle avoids bugs and allows us to update the website smoothly without any loss of functionalities.

\subsubsection*{Liskov Substitution Principle}

A subclass should behave correctly according to its parent class. For example, if we have a bird class that has method fly(), we should not create a subclass called penguin because penguins cannot fly. If we ever used a penguin to substitute for a bird, our program would break. Any subclasses of the class bird should adhere to its rules, such as a pigeon.

\subsubsection*{Interface Segregation Principle}

Smaller, more specific interfaces to avoid unnecessary methods. Our clients should only access the interfaces they need. \\

\noindent For example, our playerStats interface should only handle player statistics, while our teamStats interface should only handle team stats. This way, we do not have one big interface that handles both player and team stats, when we are only looking for one or the other.


\subsubsection*{Dependency Inversion Principle}

High-level modules should not depend on low-level modules; rather, both should rely on abstractions. This principle increases flexibility and modularity. \\

\noindent For example, our platform (high-level) should not depend directly on a specific database (low-level), such as MongoDB. Instead, there should be an abstract data layer, allowing us to swap out the database for another if needed, without impacting the rest of the system. In this way, the platform can work with a database, but the specific implementation remains flexible.

\subsection*{Responsive and Accessible Design}

Users have a variety of needs, as all users are different individuals. We want our features to be mobile-friendly and accessible. The standard here is to use ARIA roles, keyboard navigation support, semantic HTML, etc. We also do not want the colours to be negatively affecting user experience.

\subsection*{API Design \& Error Handling}

We want our design to follow REST API conventions, including descriptive endpoints that are well documented. We also need proper error handling throughout our functionalities to ensure that both the user and our system understand what error is being communicated, which will enhance the overall experience of our platform. \\

\noindent For example, if a user enters the wrong password when logging in, that message should be conveyed back to the user. This password error must also be logged into our system so that our developers can see the error.

\subsection*{Documentation}

Everyone needs to document their work, so others can understand the process that leads to the result. It is insufficient to only look at the final result because without understanding we are unable to make modifications when needed. In the engineering world, changes are bound to happen and we must document our process to reflect our understanding of that. This concept will help future developers understand and improve upon what we have built here.

\newpage{}

\section*{Appendix --- Reflection}

\begin{enumerate}
    \item Why is it important to create a development plan prior to starting the
    project?

	\begin{itemize}
		\item We believe it is important to create a development plan prior to the start of
		a project because it provide us a better idea of the scope of the project. When 
		viewing a project from an outside persepctive, it is ver easy to underestimate
		the magnitude and the responsibilities required by that project. This often results
		in a bunch of missed deadlines, increased stress throughout project duration, 
		incomplete or unreliable final product, etc.
		\item By creating a development plan, the team is able to ensure we have smaller 
		more manageable tasks, and a clear timeline for the development of the project.
		\item Developing a project plan is also very important to improve the collaboration
		and unity of the team and increase the communication at each stage of the project.
		This allows each team member to have more accountability and responsibility over
		their respective parts of the project that they are assigned.
		\item A development plan could also help us identify potential issues that we could
		run into throughout the duration of the project.
	\end{itemize}

    \item In your opinion, what are the advantages and disadvantages of using
    CI/CD?
	\subsection*{Advantages}
	\begin{itemize}
		\item A major benefit of CI/CD is streamlining our development process.
		\item By using CI/CD we are able to release code in a faster and more stable way which
		provides us with a more reliable development.
		\item Integrating CI/CD into our project gives us a huge boost in our team collaboration.
		The team is able to merge smaller pieces of code more frequently without a consistent
		need for manual intervention. This allows us to test features in smaller units which in
		return, results in a higher level of confidence in our codebase and code releases.
	\end{itemize}
	\subsection*{Disadvantages}
	\begin{itemize}
		\item Team members could sometimes be unfamiliar with CI/CD, and the changes needed to 
		adopt that skill could result in some lost productivity due to major cahnges in their
		mindset, workflow, development process, etc.
		\item There could be some distrust in the automation with the CI/CD due to unreliable
		or false test. This could result in reduced confidence in the CI/CD system and wasted time
		from debugging tests.
		\item There could also be possible security risks with using CI/CD. Deploying code with 
		project secrets or vulnerable information accidentally could expose sensitive information.
		This adds in an extra layer of consideration and consistent attention/vigilance.
	\end{itemize}
    \item What disagreements did your group have in this deliverable, if any,
    and how did you resolve them?
	\paragraph*{}
	We've had a number of disagreements so far throughout the course of this problem. Initially,
	we had a problem selecting what project to work on for the capstone. To resolve this, we had
	each member vote for their 2 top picks and after that we all discussed and we were able to
	finalize on a project.\\
	Another issue we had was with the the convention for naming branches. Each individual in the team
	has had work experience at different companies, and we learn different conventions when it comes
	to branch names. To decide on this, we took down all the different perspective and weighed each option
	based which would be most efficient and useful for the team using a matrix, then we were able
	to decide on a convention for naming branches.
\end{enumerate}

\section*{Appendix --- Team Charter}


\subsection*{External Goals}

Our team's external goals include:

\begin{itemize}
    \item Achieve a grade of 11 or greater in the capstone course.
    \item Produce a project that team members can discuss in future job interviews.
    \item Produce a project that team members can take pride in as a representation of 4 years of Software Engineering courses.
    \item Gain experience in project management and working as a cohesive team.
    \item Build strong, professional connections within the team, with faculty/supervisors.

\end{itemize}

\subsection*{Attendance}


\subsubsection*{Expectations}

Our team's expectations regarding attendance are as follows:
\begin{itemize}
    \item Team members must attend all scheduled meetings, both virtual and in-person, unless a valid reason is provided in advance.
    \item Team members are expected to arrive on time and stay for the entire duration of the meeting. Repeated lateness or early departures without prior notice or valid reasons are unacceptable.
    \item Team member that are missing a meeting must notify the team at least 24 hours in advance, or as early as possible, except in the case of emergencies.
    \item Team members should come prepared, having completed any assigned tasks and actively participating in the productivity of the meeting.
\end{itemize}

\subsubsection*{Acceptable Excuse}

Acceptable excuses for missing a meeting or arriving late:
\begin{itemize}
    \item Health-related issues.
    \item Family emergencies or urgent personal matters.
    \item Conflicts with other courses or prior academic or professional commitments.
    \item Unavoidable technical difficulties.
\end{itemize}
Unacceptable excuses:
\begin{itemize}
    \item Forgetting the meeting time or oversleeping.
    \item Non-urgent personal commitments (e.g., social events, non-critical errands).
    \item Lack of preparation for the meeting.
    \item Laziness.
\end{itemize}

\subsubsection*{In Case of Emergency}

Our teams process for emergencies:
\begin{itemize}
    \item The team member must inform the team as soon as possible through the groupchat explaining their situation.
    \item If the team member cannot attend the meeting, they should provide a brief update on their progress or any relevant information so the team can continue without them (as soon as safely and conveniently possible).
    \item In the case that a deadline cannot be met, the team member must alert the team and the team will confer on the urgency of the task and reassign tasks accordingly.
    \item For major deliverables, the team may confer to try to accommodate a reasonable extension. Team members are expected to be willing to collaborate and contribute extra work to help cover for the lost time from the emergency and put the team first.
    
\end{itemize}
In case of prolonged emergencies, the team will review the situation and may reassign responsibilities to ensure the project remains on track.


\subsection*{Accountability and Teamwork}

\subsubsection*{Quality} 

Our teams process and expectations for quality:
\begin{itemize}
    \item Each team member is expected to come fully prepared for meetings, having completed any assigned tasks and ideation processes. 
    \item Deliverables and tasks should be done to the best of the team member's abilities.
    \item At least one other team member will have reviewed each task. If a task is below the teams accepted quality, the reviewer is expected to respectfully bring it to the attention of the team and suggest improvements.
    \item Team members are expected to provide constructive criticism to other team members, and team members are expected to handle the feedback professionally. 
\end{itemize}

\subsubsection*{Attitude}

Our teams process and expectations for attitude:
\begin{itemize}
    \item Each team member is expected to act professionally and cooperatively.
    \item Team members should be expected to be willing to help out on other member's tasks, providing feedback and support (within reason).
    \item Team members should follow the code of conduct and conflict resolution plans from Harvard University:  https://hr.harvard.edu/staff-personnel-manual/employee-conduct
\end{itemize}


\subsubsection*{Stay on Track}

To keep the team on track, we will:
\begin{itemize}
    \item Have weekly progress check-ins to ensure tasks are being completed as expected.
    \item Set specific target metrics, such as attendance, commits, and task completion. 
\end{itemize}

If any team member misses a meeting without reason, they have to bring coffee at the next meeting.

\subsubsection*{Team Building}

To build team cohesion, we will organize at least 2 team-building activities over the course of the Capstone course and begin meetings with a personal check-in.

\subsubsection*{Decision Making} 

Our team will aim for decision-making by consensus, where every member's opinion is considered. In situations where consensus cannot be reached, we will vote, with the majority decision winning. For significant disputes, we will seek input from the project TA or instructor to guide the final resolution.
\end{document}