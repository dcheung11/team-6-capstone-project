\documentclass{article}

\usepackage{float}
\restylefloat{table}

\usepackage{booktabs}

\title{Team Contributions: Rev 0\\\progname}

\author{\authname}

\date{}

%% Comments

\usepackage{color}

\newif\ifcomments\commentstrue %displays comments
%\newif\ifcomments\commentsfalse %so that comments do not display

\ifcomments
\newcommand{\authornote}[3]{\textcolor{#1}{[#3 ---#2]}}
\newcommand{\todo}[1]{\textcolor{red}{[TODO: #1]}}
\else
\newcommand{\authornote}[3]{}
\newcommand{\todo}[1]{}
\fi

\newcommand{\wss}[1]{\authornote{blue}{SS}{#1}} 
\newcommand{\plt}[1]{\authornote{magenta}{TPLT}{#1}} %For explanation of the template
\newcommand{\an}[1]{\authornote{cyan}{Author}{#1}}

%% Common Parts

\newcommand{\progname}{ProgName} % PUT YOUR PROGRAM NAME HERE
\newcommand{\authname}{Team \#, Team Name
\\ Student 1 name
\\ Student 2 name
\\ Student 3 name
\\ Student 4 name} % AUTHOR NAMES                  

\usepackage{hyperref}
    \hypersetup{colorlinks=true, linkcolor=blue, citecolor=blue, filecolor=blue,
                urlcolor=blue, unicode=false}
    \urlstyle{same}
                                


\begin{document}

\maketitle

This document summarizes the contributions of each team member for the Rev 0
Demo.  The time period of interest is the time between the POC demo and the Rev
0 demo.

\section{Demo Plans}

For this demo we plan to show our finished product. This includes the full
implementation of the functioning web application with sign-ups/logins, team
formation, schedule formation, rescheduling, and everything else that
is outlined in our requirements.

\section{Team Meeting Attendance}

% \wss{For each team member how many team meetings have they attended over the
%     time period of interest.  This number should be determined from the meeting
%     issues in the team's repo.  The first entry in the table should be the total
%     number of team meetings held by the team.}

\begin{table}[H]
    \centering
    \begin{tabular}{ll}
        \toprule
        \textbf{Student}    & \textbf{Meetings} \\
        \midrule
        Total               & 8                 \\
        Damien Cheung       & 8                 \\
        Jad Haytaoglu       & 8                 \\
        Derek Li            & 8                 \\
        Temituoyo Ugborogho & 8                 \\
        Emma Wigglesworth   & 8                 \\
        \bottomrule
    \end{tabular}
\end{table}

\section{Supervisor/Stakeholder Meeting Attendance}

% \wss{For each team member how many supervisor/stakeholder team meetings have
% they attended over the time period of interest.  This number should be determined
% from the supervisor meeting issues in the team's repo.  The first entry in the
% table should be the total number of supervisor and team meetings held by the
% team.  If there is no supervisor, there will usually be meetings with
% stakeholders (potential users) that can serve a similar purpose.}

\begin{table}[H]
    \centering
    \begin{tabular}{ll}
        \toprule
        \textbf{Student}    & \textbf{Meetings} \\
        \midrule
        Total               & 3                 \\
        Damien Cheung       & 3                 \\
        Jad Haytaoglu       & 3                 \\
        Derek Li            & 3                 \\
        Temituoyo Ugborogho & 3                 \\
        Emma Wigglesworth   & 3                 \\
        \bottomrule
    \end{tabular}
\end{table}


\section{Lecture Attendance}

% \wss{For each team member how many lectures have they attended over the time
% period of interest.  This number should be determined from the lecture issues in
% the team's repo.  The first entry in the table should be the total number of
% lectures since the beginning of the term.}

\begin{table}[H]
    \centering
    \begin{tabular}{ll}
        \toprule
        \textbf{Student}    & \textbf{Lectures} \\
        \midrule
        Total               & 1                 \\
        Damien Cheung       & 1                 \\
        Jad Haytaoglu       & 1                 \\
        Derek Li            & 0                 \\
        Temituoyo Ugborogho & 1                 \\
        Emma Wigglesworth   & 1                 \\
        \bottomrule
    \end{tabular}
\end{table}

\section{TA Document Discussion Attendance}

% \wss{For each team member how many of the informal document discussion meetings
%     with the TA were attended over the time period of interest.}

\begin{table}[H]
    \centering
    \begin{tabular}{ll}
        \toprule
        \textbf{Student} & \textbf{Lectures} \\
        \midrule
        Total            & 1               \\
        Damien Cheung & 1\\
        Jad Haytaoglu & 1\\
        Derek Li & 1\\
        Temituoyo Ugborogho & 1\\
        Emma Wigglesworth & 1\\
        \bottomrule
    \end{tabular}
\end{table}

\section{Commits}

% \wss{For each team member how many commits to the main branch have been made
%     over the time period of interest.  The total is the total number of commits for
%     the entire team since the beginning of the term.  The percentage is the
%     percentage of the total commits made by each team member.}

\begin{table}[H]
    \centering
    \begin{tabular}{lll}
        \toprule
        \textbf{Student} & \textbf{Commits} & \textbf{Percent} \\
        \midrule
        Total            & 159              & 100\%            \\
        Damien Cheung          & 43              & 27\%               \\
        Jad Haytaoglu           & 34              & 21\%               \\
        Derek Li           & 24              & 15\%               \\
        Temituoyo Ugborogho           & 21              & 13\%               \\
        Emma Wigglesworth           & 37              & 23\%               \\
        \bottomrule
    \end{tabular}
\end{table}

% \wss{If needed, an explanation for the counts can be provided here.  For
%     instance, if a team member has more commits to unmerged branches, these numbers
%     can be provided here.  If multiple people contribute to a commit, git allows for
%     multi-author commits.}

\section{Issue Tracker}

\begin{table}[H]
    \centering
    \begin{tabular}{lll}
        \toprule
        \textbf{Student} & \textbf{Authored (O+C)} & \textbf{Assigned (C only)} \\
        \midrule
        Damien Cheung           & 33                     & 31                        \\
        Jad Haytaoglu           & 18                     & 17                        \\
        Derek Li           & 9                     & 9                        \\
        Temituoyo Ugborogho           & 14                     & 14                        \\
        Emma Wigglesworth           & 36                     & 29                        \\
        \bottomrule
    \end{tabular}
\end{table}

There were multiple tasks such as Figma that were not necessarily tracked
in issues.

\section{CICD}

% \wss{Say how CICD is used in your project}

% \wss{If your team has additional metrics of productivity, please feel free to
%     add them to this report.}

CI/CD is mainly used in our project to build LaTeX pdf files when .tex
files are pushed. This is because documentation has been the majority
of our contribution and importance until now. We are working on  
incorporating a code/implementation related CI/CD now.

\end{document}