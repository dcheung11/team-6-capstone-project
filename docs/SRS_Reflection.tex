\begin{enumerate}
    \item \textbf{What went well while writing this deliverable?}
    
    During the development of the SRS, the team was able to collaborate effectively, maintaining open communication and sharing ideas. The clear breakdown of tasks among team members helped streamline the process, ensuring that each section was tackled methodically. We used Google Docs to put down initial drafts and brainstorming so we could work together more easily in real time than if we were to draft using git commits. Utilizing LaTeX as the documentation format facilitated a consistent and professional layout, making it easy to organize and adjust content as needed.

    \item \textbf{What pain points did you experience during this deliverable, and how did you resolve them?}
    
    One of the major challenges faced was ensuring that all requirements were accurately captured and properly categorized, especially distinguishing between functional and non-functional requirements. To resolve this, team members would bring up to the rest of the team any concerns they had in their designated sections. Another key step was comparing the requirements document with feedback from the TA. Additionally, managing the document formatting in LaTeX was initially time-consuming, but the team grew more proficient over time through consistent practice and peer assistance. Another pain point would be the struggle to keep track of all graded content and requirements. This was because the resources we have disributed over GitLab, GitHub, and AvenueToLearn. This caused a lot of confusion for the team that we still need to get adapt to.

    \item \textbf{How many of your requirements were inspired by speaking to your client(s) or their proxies (e.g., your peers, stakeholders, potential users)?}
    
    A significant portion of the requirements—approximately 80\%—were directly inspired by interactions with the client (our project advisor). Discussion with the project advisor provided valuable insights, helping the team refine specific requirements related to user roles, scheduling, and data management.

    \item \textbf{Which of the courses you have taken, or are currently taking, will help your team to be successful with your capstone project?}
    
    The following courses are expected to be instrumental in the success of this project:
    \begin{itemize}
        \item \textbf{SFWRENG 4HC3 (Human-Computer Interfaces):} This course provides essential knowledge for designing user-friendly interfaces, critical for the web platform.
        \item \textbf{SFWRENG 4AA4 (Real-Time Systems and Control Applications):} Offers insights into designing responsive systems, which is important for building the scheduling and management components of the platform.
        \item \textbf{SFWRENG 3DB3 (Databases):} Provides the necessary skills for designing and managing databases, crucial for the project’s data management requirements.
        \item \textbf{SFWRENG 2C03 (Data Structures and Algorithms):} A foundational course that covers algorithms relevant to the scheduling functionality of the platform.
        \item \textbf{SFWRENG 3S03 (Software Testing \& Quality Assurance):} This course will aid in implementing rigorous testing methodologies to ensure that the platform meets the required quality standards.
    \end{itemize}
    
    \item \textbf{What knowledge and skills will the team collectively need to acquire to successfully complete this capstone project?} \\
    The team will need to acquire skills in several areas to ensure the successful development of the softball league platform. These include:
    \begin{itemize}
        \item \textbf{Front-End Development and User Interface Design:} Knowledge of creating user-friendly and accessible interfaces, applying Human-Computer Interfaces (HCI) principles.
        \item \textbf{Database Management:} Skills related to structuring, optimizing, and securing databases to handle large volumes of user data.
        \item \textbf{Scheduling Algorithms:} Familiarity with creating effective scheduling systems that allow team preferences and rescheduling requests to be efficiently managed.
        \item \textbf{Authentication and Security:} Understanding of secure login systems and data protection measures to ensure user information is kept safe.
        \item \textbf{System Architecture and Integration:} Skills in connecting front-end and back-end components for seamless user experience.
    \end{itemize}

    \item \textbf{For each of the knowledge areas and skills identified in the previous question, what are at least two approaches to acquiring the knowledge or mastering the skill? Of the identified approaches, which will each team member pursue, and why did they make this choice?} \\
    The approaches and team member focus areas are as follows:
    \begin{itemize}
        \item \textbf{Derek:}
        \begin{itemize}
            \item Derek will focus on enhancing front-end development and user interface design skills by applying principles from the Human-Computer Interfaces course.
            \item He will also review online tutorials on web accessibility to ensure the platform meets diverse user needs.
            \item Derek chose this approach due to his experience building web-based tools and his interest in creating an intuitive user interface that aligns with the capstone’s focus on user-friendliness.
        \end{itemize}

        \item \textbf{Emma:}
        \begin{itemize}
            \item Emma will deepen her understanding of database management by revisiting content from database courses.
            \item She will explore additional online resources focused on data optimization techniques, such as indexing and data security.
            \item Emma chose this focus because of her background in backend development and her desire to ensure data integrity and efficiency, both critical to the project’s backend development.
        \end{itemize}

        \item \textbf{Jad:}
        \begin{itemize}
            \item Jad will refine his knowledge of scheduling algorithms and system optimization by revisiting course materials from previous software engineering courses.
            \item He will also explore online tutorials on task automation to improve the efficiency of scheduling features.
            \item Jad selected this approach because of his strong analytical skills and interest in making the scheduling feature as effective as possible, which is central to the platform's functionality.
        \end{itemize}

        \item \textbf{Tuoyo:}
        \begin{itemize}
            \item Tuoyo will enhance his skills in authentication and security by reviewing past coursework related to software security practices.
            \item He will utilize online tutorials that cover modern security protocols (e.g., OAuth, JWT) to ensure secure user access control.
            \item Tuoyo selected this approach due to his interest in cybersecurity and the importance of creating a secure login system for the platform.
        \end{itemize}

        \item \textbf{Damien:}
        \begin{itemize}
            \item Damien will focus on system architecture and integration between front-end and back-end components by referring to previous coursework on software design and development.
            \item He will use online tutorials to better understand system workflows and ensure seamless component interaction.
            \item Damien chose this approach because he enjoys working on system-level design and aims to ensure the platform runs smoothly across all components.
        \end{itemize}
    \end{itemize}
\end{enumerate}