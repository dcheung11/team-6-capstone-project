\documentclass[12pt, titlepage]{article}

\usepackage{amsmath, mathtools}

\usepackage[round]{natbib}
\usepackage{amsfonts}
\usepackage{amssymb}
\usepackage{graphicx}
\usepackage{colortbl}
\usepackage{xr}
\usepackage{hyperref}
\usepackage{longtable}
\usepackage{xfrac}
\usepackage{tabularx}
\usepackage{float}
\usepackage{siunitx}
\usepackage{booktabs}
\usepackage{multirow}
\usepackage[section]{placeins}
\usepackage{caption}
\usepackage{fullpage}

\hypersetup{
bookmarks=true,     % show bookmarks bar?
colorlinks=true,       % false: boxed links; true: colored links
linkcolor=red,          % color of internal links (change box color with linkbordercolor)
citecolor=blue,      % color of links to bibliography
filecolor=magenta,  % color of file links
urlcolor=cyan          % color of external links
}

\usepackage{array}

\externaldocument{../../SRS/SRS}

%% Comments

\usepackage{color}

\newif\ifcomments\commentstrue %displays comments
%\newif\ifcomments\commentsfalse %so that comments do not display

\ifcomments
\newcommand{\authornote}[3]{\textcolor{#1}{[#3 ---#2]}}
\newcommand{\todo}[1]{\textcolor{red}{[TODO: #1]}}
\else
\newcommand{\authornote}[3]{}
\newcommand{\todo}[1]{}
\fi

\newcommand{\wss}[1]{\authornote{blue}{SS}{#1}} 
\newcommand{\plt}[1]{\authornote{magenta}{TPLT}{#1}} %For explanation of the template
\newcommand{\an}[1]{\authornote{cyan}{Author}{#1}}

%% Common Parts

\newcommand{\progname}{ProgName} % PUT YOUR PROGRAM NAME HERE
\newcommand{\authname}{Team \#, Team Name
\\ Student 1 name
\\ Student 2 name
\\ Student 3 name
\\ Student 4 name} % AUTHOR NAMES                  

\usepackage{hyperref}
    \hypersetup{colorlinks=true, linkcolor=blue, citecolor=blue, filecolor=blue,
                urlcolor=blue, unicode=false}
    \urlstyle{same}
                                


\begin{document}

\title{Module Interface Specification for \progname{}}

\author{\authname}

\date{\today}

\maketitle

\pagenumbering{roman}

\section{Revision History}

\begin{tabularx}{\textwidth}{p{3cm}p{2cm}X}
  \toprule {\bf Date} & {\bf Version} & {\bf Notes}                                  \\
  \midrule
  Jan 3               & 1.0           & Added MIS for TeamT, GameT, PlayerT, Backend \\
  Date 2              & 1.1           & Notes                                        \\
  \bottomrule
\end{tabularx}

~\newpage

\section{Symbols, Abbreviations and Acronyms}

See SRS Documentation at \wss{give url}

\wss{Also add any additional symbols, abbreviations or acronyms}

\newpage

\tableofcontents

\newpage

\pagenumbering{arabic}

\section{Introduction}

The following document details the Module Interface Specifications for
\wss{Fill in your project name and description}

Complementary documents include the System Requirement Specifications
and Module Guide.  The full documentation and implementation can be
found at \url{...}.  \wss{provide the url for your repo}

\section{Notation}

\wss{You should describe your notation.  You can use what is below as
  a starting point.}

The structure of the MIS for modules comes from \citet{HoffmanAndStrooper1995},
with the addition that template modules have been adapted from
\cite{GhezziEtAl2003}.  The mathematical notation comes from Chapter 3 of
\citet{HoffmanAndStrooper1995}.  For instance, the symbol := is used for a
multiple assignment statement and conditional rules follow the form $(c_1
  \Rightarrow r_1 | c_2 \Rightarrow r_2 | ... | c_n \Rightarrow r_n )$.

The following table summarizes the primitive data types used by \progname.

\begin{center}
  \renewcommand{\arraystretch}{1.2}
  \noindent
  \begin{tabular}{l l p{7.5cm}}
    \toprule
    \textbf{Data Type} & \textbf{Notation} & \textbf{Description}                                             \\
    \midrule
    character          & char              & a single symbol or digit                                         \\
    integer            & $\mathbb{Z}$      & a number without a fractional component in (-$\infty$, $\infty$) \\
    natural number     & $\mathbb{N}$      & a number without a fractional component in [1, $\infty$)         \\
    real               & $\mathbb{R}$      & any number in (-$\infty$, $\infty$)                              \\
    \bottomrule
  \end{tabular}
\end{center}

\noindent
The specification of \progname \ uses some derived data types: sequences, strings, and
tuples. Sequences are lists filled with elements of the same data type. Strings
are sequences of characters. Tuples contain a list of values, potentially of
different types. In addition, \progname \ uses functions, which
are defined by the data types of their inputs and outputs. Local functions are
described by giving their type signature followed by their specification.

\section{Module Decomposition}

The following table is taken directly from the Module Guide document for this project.

\begin{table}[h!]
  \centering
  \begin{tabular}{p{0.3\textwidth} p{0.6\textwidth}}
    \toprule
    \textbf{Level 1}                               & \textbf{Level 2}                \\
    \midrule

    {Hardware-Hiding}                              & ~                               \\
    \midrule

    \multirow{7}{0.3\textwidth}{Behaviour-Hiding}  & Input Parameters                \\
                                                   & Output Format                   \\
                                                   & Output Verification             \\
                                                   & Temperature ODEs                \\
                                                   & Energy Equations                \\
                                                   & Control Module                  \\
                                                   & Specification Parameters Module \\
    \midrule

    \multirow{3}{0.3\textwidth}{Software Decision} & {Sequence Data Structure}       \\
                                                   & ODE Solver                      \\
                                                   & Plotting                        \\
    \bottomrule
  \end{tabular}
  \caption{Module Hierarchy}
  \label{TblMH}
\end{table}

\newpage
~\newpage

\section{MIS of \wss{Module Name}} \label{Module} \wss{Use labels for
  cross-referencing}

\wss{You can reference SRS labels, such as R\ref{R_Inputs}.}

\wss{It is also possible to use \LaTeX for hypperlinks to external documents.}

\subsection{Module}

\wss{Short name for the module}

\subsection{Uses}


\subsection{Syntax}

\subsubsection{Exported Constants}

\subsubsection{Exported Access Programs}

\begin{center}
  \begin{tabular}{p{2cm} p{4cm} p{4cm} p{2cm}}
    \hline
    \textbf{Name}    & \textbf{In} & \textbf{Out} & \textbf{Exceptions} \\
    \hline
    \wss{accessProg} & -           & -            & -                   \\
    \hline
  \end{tabular}
\end{center}

\subsection{Semantics}

\subsubsection{State Variables}

\wss{Not all modules will have state variables.  State variables give the module
  a memory.}

\subsubsection{Environment Variables}

\wss{This section is not necessary for all modules.  Its purpose is to capture
  when the module has external interaction with the environment, such as for a
  device driver, screen interface, keyboard, file, etc.}

\subsubsection{Assumptions}

\wss{Try to minimize assumptions and anticipate programmer errors via
  exceptions, but for practical purposes assumptions are sometimes appropriate.}

\subsubsection{Access Routine Semantics}

\noindent \wss{accessProg}():
\begin{itemize}
  \item transition: \wss{if appropriate}
  \item output: \wss{if appropriate}
  \item exception: \wss{if appropriate}
\end{itemize}

\wss{A module without environment variables or state variables is unlikely to
  have a state transition.  In this case a state transition can only occur if
  the module is changing the state of another module.}

\wss{Modules rarely have both a transition and an output.  In most cases you
  will have one or the other.}

\subsubsection{Local Functions}

\wss{As appropriate} \wss{These functions are for the purpose of specification.
  They are not necessarily something that is going to be implemented
  explicitly.  Even if they are implemented, they are not exported; they only
  have local scope.}

\newpage

\bibliographystyle {plainnat}
\bibliography {../../../refs/References}

\newpage

\section{MIS of Scheduling Module} \label{SchedulingModule}

\subsection{Module}
\textbf{Scheduling Module}: Behaviour-Hiding Module

\subsection{Uses}
\begin{itemize}
    \item Scheduling Algorithm Module
    \item TeamT Module
    \item GameT Module
\end{itemize}

\subsection{Syntax}

\subsubsection{Exported Constants}
\begin{itemize}
    \item \textbf{DEFAULT\_WEEK\_COUNT: Integer} \\ Default number of weeks in the season.
\end{itemize}

\subsubsection{Exported Access Programs}
\begin{center}
  \begin{tabular}{|p{4cm}| p{4cm}| p{4cm} | p{3cm}|}
  \hline
  \textbf{Name} & \textbf{In} & \textbf{Out} & \textbf{Exceptions} \\
  \hline
  create schedule & TeamT[], Slot Object[], Integer & Schedule Object & - \\
  addGame & GameT & Boolean & InvalidInput \\
  removeGame & String & Boolean & GameNotFound \\
  updateGameSlot & String, Slot Object & Boolean & GameNotFound \\
  rescheduleGame & String, GameT, GameT & Boolean & GameNotFound, SlotNotFound \\
  \hline
  \end{tabular}
\end{center}

\subsection{Semantics}

\subsubsection{State Variables}
\begin{itemize}
    \item \textbf{schedule: Object} \\ The current schedule object containing all games, slots, and team assignments.
\end{itemize}

\subsubsection{Environment Variables}
\begin{itemize}
    \item None
\end{itemize}

\subsubsection{Assumptions}
\begin{itemize}
    \item Valid team and slot data are provided for schedule creation.
\end{itemize}

\subsubsection{Access Routine Semantics}

\noindent createSchedule(teams, slots, seasonLength):
\begin{itemize}
    \item transition: $schedule := generateSchedule(teams, slots, seasonLength)$
    \item output: $out := schedule$
    \item exception: None
\end{itemize}

\noindent addGame(details):
\begin{itemize}
    \item transition: Adds a new game to the schedule using the provided details.
    \item output: $out := true$ if the game is successfully added.
    \item exception: Raises InvalidInput if details are incomplete or invalid.
\end{itemize}

\noindent removeGame(gameId):
\begin{itemize}
    \item transition: Removes the game with the given gameId from the schedule.
    \item output: $out := true$ if the game is successfully removed.
    \item exception: Raises GameNotFound if the game does not exist.
\end{itemize}

\noindent updateGameSlot(gameId, newSlot):
\begin{itemize}
    \item transition: Updates the slot for the game with gameId in the schedule to newSlot. Removes the game from the old slot.
    \item output: $out := true$ if the update is successful.
    \item exception: Raises GameNotFound if the game does not exist.
\end{itemize}

\noindent rescheduleGame(gameId, newSlot, oldSlot):
\begin{itemize}
    \item transition: Removes the game with gameId from oldSlot and assigns it to newSlot in the schedule.
    \item output: $out := true$ if the reschedule is successful.
    \item exception: Raises GameNotFound if the game does not exist or SlotNotFound if either slot does not exist.
\end{itemize}

\subsubsection{Local Functions}
\begin{itemize}
    \item None
\end{itemize}

\newpage

\section{MIS of PlayerT Module} \label{PlayerTModule}

\subsection{Module}
\textbf{PlayerT}: Abstract Player Module.
\subsection{Uses}
TeamT
\begin{itemize}
  \item \textbf{GameT}: The Player module interacts with the Game module to track player participation in games.
  \item \textbf{TeamT}: The Player module is connected to the Team module, as players are assigned to teams.
\end{itemize}

\subsection{Syntax}

\subsubsection{Exported Constants}

\subsubsection{Exported Access Programs}

\begin{center}
  \begin{tabular}{|p{4cm}| p{4cm}| p{4cm} | p{3cm}|}
    \hline
    \textbf{Name}   & \textbf{In}                                  & \textbf{Out} & \textbf{Exceptions} \\
    \hline
    PlayerT         & String, String, String, String, Bool, String & -            & -                   \\
    getPlayerId     &                                              & String       & -                   \\
    getName         &                                              & String       & -                   \\
    getEmail        &                                              & String       & -                   \\
    getWaiverStatus &                                              & Bool         & -                   \\
    getTeam         &                                              & Bool         & -                   \\
    setWaiverStatus & Bool                                         & -            & -                   \\
    setTeam         & String                                       & -            & -                   \\
    \hline
  \end{tabular}
\end{center}

\subsection{Semantics}

\subsubsection{State Variables}

\subsubsection{Environment Variables}

// TODO: some UUID auth

\subsubsection{Assumptions}

\subsubsection{Access Routine Semantics}

\noindent PlayerT(id, n, e, p, w, t):
\begin{itemize}
  \item transition: $playerId, name, email, password, waiverStatus, team := id, n, e, p, w, t$
  \item output: $out := self$
  \item exception: None
\end{itemize}

\noindent getPlayerId():
\begin{itemize}
  \item output: $out := playerId$
  \item exception: None
\end{itemize}

\noindent getName():
\begin{itemize}
  \item output: $out := name$
  \item exception: None
\end{itemize}

\noindent getEmail():
\begin{itemize}
  \item output: $out := email$
  \item exception: None
\end{itemize}

\noindent getWaiverStatus():
\begin{itemize}
  \item output: $out := waiverStatus$
  \item exception: None
\end{itemize}

\noindent getTeam():
\begin{itemize}
  \item output: $out := team$
  \item exception: None
\end{itemize}

\noindent setTeam(t):
\begin{itemize}
  \item transition: $team := t$
  \item exception: None
\end{itemize}

\noindent setWaiverStatus(w):
\begin{itemize}
  \item transition: $waiverStatus := w$
  \item exception: None
\end{itemize}


\subsubsection{Local Functions}

\newpage

\section{MIS of GameT Module} \label{GameTModule}

\subsection{Module}
\textbf{GameT}: Abstract Game Type

\subsection{Uses}
TeamT

\subsection{Syntax}

\subsubsection{Exported Constants}

\subsubsection{Exported Access Programs}
\begin{center}
  \begin{tabular}{|p{4cm}| p{4cm}| p{4cm} | p{3cm}|}
    \hline
    \textbf{Name}  & \textbf{In}                                                    & \textbf{Out} & \textbf{Exceptions} \\
    \hline
    GameT          & String, String, Date, String, String, Integer, Integer, String & -            & -                   \\
    getGameId      &                                                                & String       & -                   \\
    getTeamsInGame &                                                                & TeamT[]      & GameNotFound        \\
    getGameDetails &                                                                & Object       & GameNotFound        \\
    getStatus      &                                                                & String       & GameNotFound        \\
    getField       &                                                                & String       & GameNotFound        \\
    setStatus      & String                                                         & -            & InvalidStatus       \\
    setField       & String                                                         & -            & InvalidField        \\
    setScore       & Integer, Integer                                               & -            & InvalidScore        \\
    \hline
  \end{tabular}
\end{center}

\subsection{Semantics}

\subsubsection{State Variables}

\subsubsection{Environment Variables}
None

\subsubsection{Assumptions}
\begin{itemize}
  \item A game must have two teams assigned to it.
  \item A game must have a field assigned to it.
  \item A game's score can only be updated after the game is completed.
  \item The game status must be updated to reflect its current state (e.g., scheduled, completed).
\end{itemize}

\subsubsection{Access Routine Semantics}

\noindent GameT(id, t1, t2, d, t, s1, s2, f):
\begin{itemize}
  \item transition: $gameId, team1Id, team2Id, gameDate, gameTime, scoreTeam1, scoreTeam2, field := id, t1, t2, d, t, s1, s2, f$
  \item output: $out := self$
  \item exception: None
\end{itemize}

\noindent getGameId():
\begin{itemize}
  \item output: $out := gameId$
  \item exception: None
\end{itemize}

\noindent getGameDetails():
\begin{itemize}
  \item output: $out := \text{Object containing game details: } gameId, team1Id, team2Id, gameDate, gameTime, scoreTeam1, scoreTeam2, status, field$
  \item exception: Game not found if the game ID does not exist.
\end{itemize}

\noindent getTeamsInGame():
\begin{itemize}
  \item output: $out := \text{Array of Teams participating in the game}$
  \item exception: Game not found if the game ID does not exist.
\end{itemize}

\noindent getStatus():
\begin{itemize}
  \item output: $out := status$
  \item exception: Game not found if the game ID does not exist.
\end{itemize}

\noindent getField():
\begin{itemize}
  \item output: $out := field$
  \item exception: Game not found if the game ID does not exist.
\end{itemize}

\noindent setStatus(status):
\begin{itemize}
  \item transition: $status := status$
  \item exception: Invalid status if the provided status is not valid.
\end{itemize}

\noindent setField(field):
\begin{itemize}
  \item transition: $field := field$
  \item exception: Invalid field if the provided field is not valid.
\end{itemize}

\noindent setScore(score1, score2):
\begin{itemize}
  \item transition: $scoreTeam1 := score1, scoreTeam2 := score2$
  \item exception: Invalid score if the provided scores are not valid integers.
\end{itemize}

\subsubsection{Local Functions}
\begin{itemize}
  \item \textbf{validateGameDetails()}: A function to validate the input details when creating a new game.
\end{itemize}

\section{MIS of TeamT Module} \label{TeamModule}

\subsection{Module}
\textbf{TeamT Module}: Abstract Team Module

\subsection{Uses}
PlayerT, GameT

\subsection{Syntax}

\subsubsection{Exported Constants}

\subsubsection{Exported Access Programs}

\begin{center}
  \begin{tabular}{|p{4cm}| p{4cm}| p{4cm} | p{3cm}|}
    \hline
    \textbf{Name} & \textbf{In}                                & \textbf{Out} & \textbf{Exceptions} \\
    \hline
    TeamT         & String, String, String, PlayerT[], PlayerT & -            & -                   \\
    getTeamId     &                                            & String       & -                   \\
    getTeamName   &                                            & String       & -                   \\
    getDivision   &                                            & String       & -                   \\
    getRoster     &                                            & PlayerT[]    & -                   \\
    getCaptain    &                                            & PlayerT      & -                   \\
    addPlayer     & PlayerT                                    & -            & InvalidPlayer       \\
    removePlayer  & PlayerT                                    & -            & PlayerNotFound      \\
    \hline
  \end{tabular}
\end{center}

\subsection{Semantics}

\subsubsection{State Variables}

\subsubsection{Environment Variables}
None

\subsubsection{Assumptions}
\begin{itemize}
  \item Each team must have a unique team ID.
  \item A team can belong to one division at a time.
  \item The team roster must be an array or list of players (with unique player identifiers).
\end{itemize}

\subsubsection{Access Routine Semantics}

\noindent TeamT(id, n, d, r, c):
\begin{itemize}
  \item transition: $teamId, teamName, division, roster, captain := id, n, d, r, c$
  \item output: $out := self$
  \item exception: None
\end{itemize}

\noindent getTeamId():
\begin{itemize}
  \item output: $out := teamId$
  \item exception: None
\end{itemize}

\noindent getTeamName():
\begin{itemize}
  \item output: $out := teamName$
  \item exception: None
\end{itemize}

\noindent getDivision():
\begin{itemize}
  \item output: $out := division$
  \item exception: None
\end{itemize}

\noindent getRoster():
\begin{itemize}
  \item output: $out := roster$
  \item exception: None
\end{itemize}

\noindent getCaptain():
\begin{itemize}
  \item output: $out := captain$
  \item exception: None
\end{itemize}

\noindent addPlayer(player):
\begin{itemize}
  \item transition: $roster := roster + player$
  \item exception: Invalid player if the player is invalid or already in the roster.
\end{itemize}

\noindent removePlayer(player):
\begin{itemize}
  \item transition: $roster := roster - player$
  \item exception: Player not found if the player is not in the roster.
\end{itemize}

\subsubsection{Local Functions}
None

\section{MIS of Backend Module} \label{Backend}

\subsection{Module}

Backend/Database

\subsection{Uses}
PlayerT, GameT, TeamT

\subsection{Syntax}

\subsubsection{Exported Constants}

\subsubsection{Exported Access Programs}

\begin{center}
  \begin{tabular}{|p{4cm}| p{4cm}| p{4cm} | p{3cm}|}
    \hline
    \textbf{Name}        & \textbf{In}                                                    & \textbf{Out} & \textbf{Exceptions} \\
    \hline
    createPlayer         & String, String, String, String, Bool, String                   & -            & PlayerCreationError \\
    getPlayer            & String                                                         & PlayerT      & PlayerNotFound      \\
    updatePlayer         & String, String, String, String, Bool, String                   & -            & PlayerNotFound      \\
    deletePlayer         & String                                                         & -            & PlayerNotFound      \\
    createTeam           & String, String, String, PlayerT[], PlayerT                     & -            & TeamCreationError   \\
    getTeam              & String                                                         & TeamT        & TeamNotFound        \\
    updateTeam           & String, String, String, PlayerT[], PlayerT                     & -            & TeamCreationError   \\
    deleteTeam           & String                                                         & -            & TeamNotFound        \\
    createGame           & String, String, Date, String, String, Integer, Integer, String & -            & GameCreationError   \\
    getGame              & String                                                         & GameT        & GameNotFound        \\
    updateGame           & String, String, Date, String, String, Integer, Integer, String & -            & GameCreationError   \\
    deleteGame           & String                                                         & -            & GameNotFound        \\
    getAllPlayersForTeam & String                                                         & PlayerT[]    & TeamNotFound        \\
    getAllGamesForTeam   & String                                                         & GameT[]      & TeamNotFound        \\
    \hline
  \end{tabular}
\end{center}

\subsection{Semantics}

\subsubsection{State Variables}

\wss{Not all modules will have state variables.  State variables give the module
  a memory.}

\subsubsection{Environment Variables}

\wss{This section is not necessary for all modules.  Its purpose is to capture
  when the module has external interaction with the environment, such as for a
  device driver, screen interface, keyboard, file, etc.}

\subsubsection{Assumptions}

\begin{itemize}
  \item The backend is connected to a database
  \item Each database operation (CRUD) will be encapsulated in a backend method to ensure separation of concerns
  \item Data consistency and integrity are maintained by the backend during each operation.

\end{itemize}

\subsubsection{Access Routine Semantics}

\noindent createPlayer(name, email, password, waiverStatus, team):
\begin{itemize}
  \item transition: $playerId, name, email, password, waiverStatus, team := name, email, password, waiverStatus, team$
  \item exception: "Player Creation Error" if player cannot be created
\end{itemize}

\noindent getPlayer(playerId):
\begin{itemize}
  \item output: $out := player$ (retrieves player object based on playerId)
  \item exception: "Player Not Found" if player does not exist
\end{itemize}

\noindent updatePlayer(playerId, name, email, password, waiverStatus, team):
\begin{itemize}
  \item transition: $name, email, password, waiverStatus, team := name, email, password, waiverStatus, team$ (updates player's details)
  \item exception: "Player Not Found" if player does not exist
\end{itemize}

\noindent deletePlayer(playerId):
\begin{itemize}
  \item transition: $player := null$ (deletes the player from the database)
  \item exception: "Player Not Found" if player does not exist
\end{itemize}

\noindent createTeam(teamName, division, captain, roster):
\begin{itemize}
  \item transition: $teamId, teamName, division, captain, roster := teamName, division, captain, roster$
  \item exception: "Team Creation Error" if team cannot be created
\end{itemize}

\noindent getTeam(teamId):
\begin{itemize}
  \item output: $out := team$ (retrieves team object based on teamId)
  \item exception: "Team Not Found" if team does not exist
\end{itemize}

\noindent updateTeam(teamId, teamName, division, roster):
\begin{itemize}
  \item transition: $teamName, division, roster := teamName, division, roster$ (updates team's details)
  \item exception: "Team Not Found" if team does not exist
\end{itemize}

\noindent deleteTeam(teamId):
\begin{itemize}
  \item transition: $team := null$ (deletes the team from the database)
  \item exception: "Team Not Found" if team does not exist
\end{itemize}

\noindent createGame(teams, date, time, field, score):
\begin{itemize}
  \item transition: $gameId, teams, date, time, field, score := teams, date, time, field, score$ (creates a new game)
  \item exception: "Game Creation Error" if game cannot be created (e.g., scheduling conflict)
\end{itemize}

\noindent getGame(gameId):
\begin{itemize}
  \item output: $out := game$ (retrieves game object based on gameId)
  \item exception: "Game Not Found" if game does not exist
\end{itemize}

\noindent updateGame(gameId, updates):
\begin{itemize}
  \item transition: $gameId, updates := gameId, updates$ (updates the game's details)
  \item exception: "Game Not Found" if game does not exist
\end{itemize}

\noindent deleteGame(gameId):
\begin{itemize}
  \item transition: $game := null$ (deletes the game from the database)
  \item exception: "Game Not Found" if game does not exist
\end{itemize}

\noindent getAllPlayersForTeam(teamId):
\begin{itemize}
  \item output: $out := players$ (retrieves all players associated with the team)
  \item exception: "Team Not Found" if team does not exist
\end{itemize}

\noindent getAllGamesForTeam(teamId):
\begin{itemize}
  \item output: $out := games$ (retrieves all games associated with the team)
  \item exception: "Team Not Found" if team does not exist
\end{itemize}

\subsubsection{Local Functions}

\section{MIS of Notification Module} \label{Module:Notification}

\subsection{Module}

Notification

\subsection{Uses}

\begin{itemize}
    \item None
\end{itemize}

\subsection{Syntax}

\subsubsection{Exported Constants}

\begin{itemize}
    \item MAX\_NOTIFICATION\_LENGTH: Maximum allowed characters in a notification message.
    \item NOTIFICATION\_RETRY\_LIMIT: Maximum retry attempts for failed notifications.
\end{itemize}

\subsubsection{Exported Access Programs}

\begin{center}
\begin{tabular}{p{2cm} p{4cm} p{4cm} p{2cm}}
\hline
\textbf{Name} & \textbf{In} & \textbf{Out} & \textbf{Exceptions} \\
\hline
\texttt{send} & NotificationID, playerID & Boolean & InvalidPlayerID, SendFailure \\
\hline
\texttt{schedule} & NotificationID, DateTime & Boolean & InvalidDateTime \\
\hline
\texttt{status} & NotificationID & Status & InvalidNotificationID \\
\hline
\end{tabular}
\end{center}

\subsection{Semantics}

\subsubsection{State Variables}

\begin{itemize}
    \item \texttt{pendingNotifications}: List of notifications yet to be delivered.
    \item \texttt{deliveredNotifications}: List of successfully delivered notifications.
\end{itemize}

\subsubsection{Environment Variables}

\begin{itemize}
    \item Email Gateway: For sending email notifications.
    \item SMS Gateway: For sending SMS notifications.
\end{itemize}

\subsubsection{Assumptions}

\begin{itemize}
    \item Users have valid email addresses or phone numbers stored in the database.
    \item Gateway APIs are operational and accessible.
\end{itemize}

\subsubsection{Access Routine Semantics}

\noindent \texttt{send}(NotificationID, playerID):
\begin{itemize}
    \item transition: Moves the notification from \texttt{pendingNotifications} to \texttt{deliveredNotifications} if successfully sent.
    \item output: \texttt{true} if the notification is successfully sent; \texttt{false} otherwise.
    \item exception: \texttt{InvalidPlayerID} if the playerID is not found; \texttt{SendFailure} if the notification fails to send after retries.
\end{itemize}

\noindent \texttt{schedule}(NotificationID, DateTime):
\begin{itemize}
    \item transition: Adds the notification to the \texttt{pendingNotifications} queue with the scheduled delivery time.
    \item output: \texttt{true} if the scheduling is successful; \texttt{false} otherwise.
    \item exception: \texttt{InvalidDateTime} if the DateTime is in the past or improperly formatted.
\end{itemize}

\noindent \texttt{status}(NotificationID):
\begin{itemize}
    \item transition: None.
    \item output: The status of the notification (e.g., Pending, Delivered, Failed).
    \item exception: \texttt{InvalidNotificationID} if the NotificationID is not found.
\end{itemize}

\subsubsection{Local Functions}

\begin{itemize}
    \item \texttt{validateNotification(NotificationID)}: Ensures the notification exists and is properly formatted.
    \item \texttt{retryFailedNotifications()}: Attempts to resend notifications marked as failed.
\end{itemize}

\newpage

\section{MIS of Authentication Module} \label{Auth}

\subsection{Module}

\textbf{Auth} (M2) - Abstract object for handling user authentication and session management.

\subsection{Uses}

\begin{itemize}
	\item \textbf{User Interface Module} (M1) - For collecting user credentials (username, password)
	and displaying authentication feedback.
	\item \textbf{Backend Module} (M13) - For verifying credentials, managing tokens, and storing
	authentication data.
\end{itemize}


\subsection{Syntax}

\subsubsection{Exported Constants}
\begin{itemize}
	\item \texttt{SESSION\_TIMEOUT} - Integer representing session timeout in minutes (default: 30).
\end{itemize}

\subsubsection{Exported Access Programs}

\begin{center}
\begin{tabular}{|p{3cm} | p{4cm} | p{4cm} | p{4cm}|}
\hline
\textbf{Name} & \textbf{In} & \textbf{Out} & \textbf{Exceptions} \\
\hline
\texttt{login} & String, String & Boolean & \texttt{InvalidCredentials} \\
\texttt{logout} & String & Void & \texttt{SessionNotFound} \\
\texttt{registerUser} & String, String, String & Boolean & \texttt{DuplicateUserError} \\
\texttt{verifyToken} & String & Boolean & \texttt{TokenExpiredError} \\
\texttt{generateToken} & String & String (Token) & \texttt{UserNotFound} \\
\hline
\end{tabular}
\end{center}

\subsection{Semantics}

\subsubsection{State Variables}

\begin{itemize}
    \item \textbf{userSessions}: Map of active session tokens to user IDs.
    \item \textbf{userData}: Map of user IDs to credentials and roles.
\end{itemize}

\subsubsection{Environment Variables}

\begin{itemize}
    \item \textbf{Database Connection}: Used for storing and retrieving user authentication data.
    \item \textbf{SSL/TLS Connection}: Required for secure communication between client and server.
\end{itemize}

\subsubsection{Assumptions}

\begin{itemize}
    \item All passwords are stored as securely hashed values.
    \item Token expiration is managed based on \texttt{SESSION\_TIMEOUT}.
\end{itemize}

\subsubsection{Access Routine Semantics}

\begin{itemize}
    \item \texttt{login(username, password)}
    \begin{itemize}
        \item \textbf{Transition}: If the username and password match, generate a session token and add it to \texttt{userSessions}.
        \item \textbf{Output}: Returns \texttt{true} if successful, otherwise throws \texttt{InvalidCredentials}.
        \item \textbf{Exception}: Throws \texttt{InvalidCredentials} if the username or password is incorrect.
    \end{itemize}

    \item \texttt{logout(token)}
    \begin{itemize}
        \item \textbf{Transition}: Removes the token from \texttt{userSessions}.
        \item \textbf{Output}: None.
        \item \textbf{Exception}: Throws \texttt{SessionNotFound} if the token is invalid or expired.
    \end{itemize}

    \item \texttt{registerUser(username, password, role)}
    \begin{itemize}
        \item \textbf{Transition}: Adds a new entry to \texttt{userData} with hashed password and role.
        \item \textbf{Output}: Returns \texttt{true} if registration is successful.
        \item \textbf{Exception}: Throws \texttt{DuplicateUserError} if the username already exists.
    \end{itemize}

    \item \texttt{verifyToken(token)}
    \begin{itemize}
        \item \textbf{Output}: Returns \texttt{true} if the token is valid, otherwise throws \texttt{TokenExpiredError}.
        \item \textbf{Exception}: Throws \texttt{TokenExpiredError} if the token is expired.
    \end{itemize}

    \item \texttt{generateToken(userID)}
    \begin{itemize}
        \item \textbf{Output}: Generates a unique token linked to the \texttt{userID}.
        \item \textbf{Exception}: Throws \texttt{UserNotFound} if the user ID does not exist.
    \end{itemize}
\end{itemize}

\subsubsection{Local Functions}

\begin{itemize}
    \item \texttt{hashPassword(password)}: Converts a plaintext password into a securely hashed value.
    \item \texttt{validatePassword(inputPassword, storedHash)}: Compares an input password to the stored hashed password.
    \item \texttt{generateUniqueToken(userID)}: Generates a cryptographically secure token linked to a user ID.
\end{itemize}

\section{MIS of Announcements Module} \label{AnnouncementsModule}

\subsection{Module}
Announcements

\subsection{Uses}
\begin{itemize}
    \item Notification Module
    \item Backend Module
    
\subsection{Syntax}

\subsubsection{Exported Constants}
  \item \texttt{ANNOUNCEMENT\_TYPE\_INFO}: Constant for an informational announcement type.
\end{itemize}

\subsubsection{Exported Access Programs}
  \begin{center}
  \begin{tabular}{|p{3.2cm}|p{4cm}|p{3.5cm}|p{4cm}|}
  \hline
  \textbf{Name} & \textbf{In} & \textbf{Out} & \textbf{Exceptions} \\
  \hline
  \texttt{sendAnnouncement} & Announcement details (text, type) & Announcement ID & InvalidInput \\
  \texttt{updateAnnouncement} & Announcement ID & Boolean & AnnouncementNotFound, InvalidUpdate \\
  \texttt{deleteAnnouncement} & Announcement ID & Boolean & AnnouncementNotFound \\
  \hline
  \end{tabular}
\end{center}

\subsection{Semantics}

\subsubsection{State Variables}
\begin{itemize}
    \item None
\end{itemize}

\subsubsection{Environment Variables}
\begin{itemize}
    \item None
\end{itemize}

\subsubsection{Assumptions}
\begin{itemize}
    \item All announcement operations (create, update, delete) are performed by authorized users only.
    \item Users will be notified of new or updated announcements based on their notification preferences.
    
\subsubsection{Access Routine Semantics}

\noindent \texttt{sendAnnouncement}(announcementID):  
\begin{itemize}
    \item transition: Sends a notification about the specified announcement to subscribed users using send()
    \item output: Returns \texttt{true} if the notification was sent successfully.
    \item exception: Raises \texttt{AnnouncementNotFoundException} if the announcement does not exist.
\end{itemize}

\noindent \texttt{updateAnnouncement}(announcementID):  
\begin{itemize}
    \item transition: Updates the details of the specified announcement
    \item output: Returns \texttt{true} if the update is successful.
    \item exception: Raises \texttt{AnnouncementNotFound} if the announcement does not exist
\end{itemize}

\noindent \texttt{deleteAnnouncement}(announcementID):  
\begin{itemize}
    \item transition: Removes the specified announcement
    \item output: Returns \texttt{true} if the deletion is successful.
    \item exception: Raises \texttt{AnnouncementNotFound} if the announcementID does not exist.
\end{itemize}

\subsubsection{Local Functions}
\begin{itemize}
    \item None
\end{itemize}

\newpage

\section{MIS of Scheduling Algorithm Module} \label{SchedulingAlgorithmModule}

\subsection{Module}
SchedulingAlgorithm: Software Decision Module.

\subsection{Uses}
\begin{itemize}
  \item Scheduling Module
  \item TeamT
  \item SlotT
  \item GameT
\end{itemize}

\subsection{Syntax}

\subsubsection{Exported Constants}
\begin{itemize}
  \item DEFAULT\_WEEK\_COUNT: Integer \\ Default number of weeks for scheduling.
  \item MAX\_GAMES\_PER\_TEAM: Integer \\ Maximum number of games per team.
  \item MIN\_GAMES\_PER\_TEAM: Integer \\ Minimum number of games per team.
\end{itemize}

\subsubsection{Exported Access Programs}

\begin{center}
  \begin{tabular}{|p{3cm}|p{3.5cm}|p{3.5cm}|p{5cm}|}
  \hline
  Name & In & Out & Exceptions \\
  \hline
  generateSchedule & TeamT[], Slot Object[] & Schedule object & InvalidInput \\
  resolveConflicts & Schedule Object & Schedule Object & ConflictResolutionFailure \\
  optimizeSchedule & Schedule Object & Schedule Object & OptimizationFailure \\
  \hline
  \end{tabular}
\end{center}

\subsection{Semantics}

\subsubsection{State Variables}
None

\subsubsection{Environment Variables}
None

\subsubsection{Assumptions}
\begin{itemize}
  \item Inputs, such as the list of teams, slots, and week count, are valid and non-empty.
  \item Teams have all required properties, such as constraints and preferences.
  \item Slots are pre-validated and adhere to game capacity limits.
  \item Conflicts are identifiable based on provided rules (e.g., double bookings, unavailable fields).
\end{itemize}

\subsubsection{Access Routine Semantics}

\noindent generateSchedule(teams, slots, weekCount):
\begin{itemize}
  \item output: out := schedule \\ A schedule object generated based on team constraints, slot availability, and the given week count.
  \item exception: Raises InvalidInput if the input is incomplete or invalid.
\end{itemize}

\noindent resolveConflicts(schedule):
\begin{itemize}
  \item transition: Resolves scheduling conflicts.
  \item output: out := updated\_schedule.
  \item exception: Raises ConflictResolutionFailure if the conflicts cannot be resolved.
\end{itemize}

\noindent optimizeSchedule(schedule):
\begin{itemize}
  \item transition: Refines the input schedule to improve adherence to the specified metrics.
  \item output: out := optimized\_schedule.
  \item exception: Raises OptimizationFailure if optimization fails due to constraints or conflicts.
\end{itemize}

\subsubsection{Local Functions}
\begin{itemize}
  \item calculateFairness(schedule): Computes a fairness metric for the schedule, ensuring balance across teams.
  \item detectConflicts(schedule): Identifies conflicts, such as double bookings or unavailable slots, in the schedule.
  \item applyOptimization(schedule, metrics): Applies optimization techniques to enhance the schedule.
\end{itemize}

\newpage

\section{MIS of Game Management Module (M4)} \label{GameManagementModule}

\subsection{Module}

Game Management Module

\subsection{Uses}

GameT Module, TeamT Module, Scheduling Module, and Database Module
\subsection{Syntax}

\subsubsection{Exported Constants}

None

\subsubsection{Exported Access Programs}

\begin{center}
  \begin{tabularx}{\textwidth}{|l|X|X|X|}
    \hline
    \textbf{Name}    & \textbf{In}   & \textbf{Out} & \textbf{Exceptions}          \\
    \hline
    reportScore      & int, int, int & -     & InvalidGameID, InvalidScore  \\
    \hline
    updateGameStatus & int, string   & -      & InvalidGameID, InvalidStatus \\
    \hline
    getGameDetails   & int           & GameT        & InvalidGameID                \\
    \hline
  \end{tabularx}
\end{center}



\subsection{Semantics}

\subsubsection{State Variables}

\begin{itemize}
  \item \texttt{games}: A collection of all GameT objects which contain game details including game ID, participating teams, scores, and stasuses
\end{itemize}

\subsubsection{Environment Variables}

\begin{itemize}
  \item Database: For storing and retrieving game records.
  \item User Interface Module: For allowing users to report scores and update statuses.
\end{itemize}

\subsubsection{Assumptions}

\begin{itemize}
  \item Game IDs are unique and valid.
  \item Scores are reported accurately and honestly
  \item Game statuses are reported or updated accurately by only captains or administrators
\end{itemize}

\subsubsection{Access Routine Semantics}

\noindent \texttt{reportScore}(gameID, team1Score, team2Score):
\begin{itemize}
  \item transition: game.score1, game.score2 := team1Score, team2Score
  \item exception: Throws \texttt{InvalidGameIDException} if the game does not exist. Throws \texttt{InvalidScoreException} if the scores are invalid.
\end{itemize}

\noindent \texttt{updateGameStatus}(gameID, newStatus):
\begin{itemize}
  \item transition: game.status := newStatus
  \item exception: Throws \texttt{InvalidGameIDException} if the game does not exist. Throws \texttt{InvalidStatusException} if the status is invalid.
\end{itemize}

\noindent \texttt{getGameResults}(gameID):
\begin{itemize}
  \item output: out := game (retrieves GameT object based on gameId)
  \item exception: Throws \texttt{InvalidGameIDException} if the game does not exist.
\end{itemize}

\subsubsection{Local Functions}

None

\newpage

\section{Appendix} \label{Appendix}

\wss{Extra information if required}

\newpage{}

\section*{Appendix --- Reflection}

\wss{Not required for CAS 741 projects}

The information in this section will be used to evaluate the team members on the
graduate attribute of Problem Analysis and Design.

The purpose of reflection questions is to give you a chance to assess your own
learning and that of your group as a whole, and to find ways to improve in the
future. Reflection is an important part of the learning process.  Reflection is
also an essential component of a successful software development process.  

Reflections are most interesting and useful when they're honest, even if the
stories they tell are imperfect. You will be marked based on your depth of
thought and analysis, and not based on the content of the reflections
themselves. Thus, for full marks we encourage you to answer openly and honestly
and to avoid simply writing ``what you think the evaluator wants to hear.''

Please answer the following questions.  Some questions can be answered on the
team level, but where appropriate, each team member should write their own
response:


\begin{enumerate}
  \item What went well while writing this deliverable?
  \item What pain points did you experience during this deliverable, and how
        did you resolve them?
  \item Which of your design decisions stemmed from speaking to your client(s)
        or a proxy (e.g. your peers, stakeholders, potential users)? For those that
        were not, why, and where did they come from?
  \item While creating the design doc, what parts of your other documents (e.g.
        requirements, hazard analysis, etc), it any, needed to be changed, and why?
  \item What are the limitations of your solution?  Put another way, given
        unlimited resources, what could you do to make the project better? (LO\_ProbSolutions)
  \item Give a brief overview of other design solutions you considered.  What
        are the benefits and tradeoffs of those other designs compared with the chosen
        design?  From all the potential options, why did you select the documented design?
        (LO\_Explores)
\end{enumerate}


\end{document}