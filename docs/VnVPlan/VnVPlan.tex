\documentclass[12pt, titlepage]{article}

\usepackage{booktabs}
\usepackage{tabularx}
\usepackage{makecell}
\usepackage{hyperref}
\hypersetup{
    colorlinks,
    citecolor=blue,
    filecolor=black,
    linkcolor=red,
    urlcolor=blue
}
\usepackage[round]{natbib}
\usepackage{longtable}

%% Comments

\usepackage{color}

\newif\ifcomments\commentstrue %displays comments
%\newif\ifcomments\commentsfalse %so that comments do not display

\ifcomments
\newcommand{\authornote}[3]{\textcolor{#1}{[#3 ---#2]}}
\newcommand{\todo}[1]{\textcolor{red}{[TODO: #1]}}
\else
\newcommand{\authornote}[3]{}
\newcommand{\todo}[1]{}
\fi

\newcommand{\wss}[1]{\authornote{blue}{SS}{#1}} 
\newcommand{\plt}[1]{\authornote{magenta}{TPLT}{#1}} %For explanation of the template
\newcommand{\an}[1]{\authornote{cyan}{Author}{#1}}

%% Common Parts

\newcommand{\progname}{ProgName} % PUT YOUR PROGRAM NAME HERE
\newcommand{\authname}{Team \#, Team Name
\\ Student 1 name
\\ Student 2 name
\\ Student 3 name
\\ Student 4 name} % AUTHOR NAMES                  

\usepackage{hyperref}
    \hypersetup{colorlinks=true, linkcolor=blue, citecolor=blue, filecolor=blue,
                urlcolor=blue, unicode=false}
    \urlstyle{same}
                                


\begin{document}

\title{System Verification and Validation Plan for \progname{}} 
\author{\authname}
\date{\today}
	
\maketitle

\pagenumbering{roman}

\section*{Revision History}

\begin{tabularx}{\textwidth}{p{3cm}p{2cm}X}
\toprule {\bf Date} & {\bf Version} & {\bf Notes}\\
\midrule
Oct 30, 2024 1 & 1.0 & 4.2 and Appendix\\

\bottomrule
\end{tabularx}

~\\
\wss{The intention of the VnV plan is to increase confidence in the software.
However, this does not mean listing every verification and validation technique
that has ever been devised.  The VnV plan should also be a \textbf{feasible}
plan. Execution of the plan should be possible with the time and team available.
If the full plan cannot be completed during the time available, it can either be
modified to ``fake it'', or a better solution is to add a section describing
what work has been completed and what work is still planned for the future.}

\wss{The VnV plan is typically started after the requirements stage, but before
the design stage.  This means that the sections related to unit testing cannot
initially be completed.  The sections will be filled in after the design stage
is complete.  the final version of the VnV plan should have all sections filled
in.}

\newpage

\tableofcontents

\listoftables
\wss{Remove this section if it isn't needed}

\listoffigures
\wss{Remove this section if it isn't needed}

\newpage

\section{Symbols, Abbreviations, and Acronyms}

\renewcommand{\arraystretch}{1.2}
\begin{tabular}{l l} 
  \toprule		
  \textbf{symbol} & \textbf{description}\\
  \midrule 
  RBAC & Role-Based Access Control. Used for defining user permissions\\
  UI & User Interface\\
  JWT & JSON Web Token. A compact way to transmit info securely\\
  GSA & Graduate Students' Association. The association overseeing the league\\
  SRS & Software Requirements Specification \\
  MIS & Module Interface Specification \\
  DD & Design Document \\
  VnV & Verification and Validation \\
  CI & Continuous Integration \\
  \bottomrule
\end{tabular}\\

\newpage

\pagenumbering{arabic}

This document defines the procedures, testing strategies, and methodologies that will be used to ensure that the platform meets its functional, performance, and user experience requirements. Verification and Validation are essential to establishing that the system is developed in accordance with its specifications and operates reliably within the designated environment.
Roadmap of the VnV document:
\begin{itemize}
  \item \textbf{General Information:} Provides a summary, objectives, challenges, and relevant documentation for the Verification and Validation plan.
  \item \textbf{Plan:} Outlines the team structure, testing plans for the requirements and design, automated testing tools, and the software validation plan.
  \item \textbf{System Tests:} Describes the system-level test cases for functional and nonfunctional requirements, including detailed tests for authentication, team management, scheduling, and additional aspects such as usability, performance, and security. It also includes a traceability matrix that maps test cases to requirements.
  \item \textbf{Unit Test Description:} Specifies the scope of unit testing, with detailed cases for both functional and nonfunctional requirements. Traceability between test cases and modules is also included here.
\end{itemize}

\section{General Information}

\subsection{Summary}

The software being tested is the McMaster GSA Softball League Management Platform. This platform is designed to support and streamline operations for the McMaster GSA softball league. Key features include team management, game scheduling, reporting functionalities, and communication tools. The platform aims to simplify tasks for league administrators, captains, and players by providing a centralized system for managing league information, tracking standings, scheduling games, and maintaining up-to-date team rosters.

\subsection{Objectives}

\subsubsection*{Primary}
\begin{itemize}
    \item \textbf{Build confidence in software correctness}: Ensure that core functionalities, including team management, game scheduling, and result reporting, operate as expected.
    \item \textbf{Demonstrate adequate usability}: Confirm that the system interface is intuitive and accessible for users with varied technical backgrounds, such as league commissioners, captains, and players.
    \item \textbf{Verify data integrity}: Ensure accurate and consistent data across all features, from player rosters to game results and standings.
    \item \textbf{Assess security}: Validate that user access levels and authentication are correctly implemented to secure sensitive data.
\end{itemize}
\subsubsection*{Out of Scope}
\begin{itemize}
    \item \textbf{Accessibility Testing}: Accessibility checks, such as screen reader compatibility, keyboard-only navigation, and compliance with accessibility standards (e.g., WCAG), will not be covered in this scope. This decision is made because the platform’s primary user base does not require stringent accessibility support, and our focus is primarily on basic usability rather than comprehensive accessibility.
    
    \item \textbf{Performance and Load Testing}: Extensive load and stress testing to measure system performance under high user traffic will be omitted. Given the platform's limited expected user base and the controlled environment in which it will operate, performance considerations are minimal. Our team will assume that system responsiveness and speed are sufficient for a small to medium load without simulating peak traffic.
    
    \item \textbf{In-depth Security and Vulnerability Testing}: While security basics, such as user authentication and access control, will be verified, extensive security testing (e.g., penetration testing and vulnerability scanning) is outside our scope. This is due to resource constraints and the relatively low-risk nature of the application, which handles limited sensitive data and is intended for a small group of users within the league.
    
    \item \textbf{External Library and API Validation}: It is assumed that all libraries and APIs have been thoroughly tested by their developers, so they will not undergo additional validation in our testing scope. Our focus will be on integration rather than on validating the correctness of external components themselves.
\end{itemize}


\subsection{Challenge Level and Extras}

The challenge level we have established for this project is General.
The extras we have described for our project include usability testing, user
documentation, and design thinking. We believe these are strongly relevant
to our project since it is a heavily user-centric web application that must in-
corporate many principles of human-computer interfacing. We will track our
changes consistently with written documentation as well as version control (via
Git and GitHub). We will incorporate design thinking when developing and
designing our system and UI, and we will conduct comprehensive usability tests
with existing/potential stakeholders to compile improvements and verify design
decisions.

\subsection{Relevant Documentation}

\begin{itemize}
  \item \textbf{Software Requirements Specification (SRS)} \citep{SRS}: The SRS defines the detailed functional and non-functional requirements of the system, providing a clear basis for testing each requirement. By mapping each test case back to the SRS, we can ensure complete coverage of all system specifications. 

  \item \textbf{Hazard Analysis} \citep{Hazard-Analysis}: The Hazard Analysis identifies potential hazards associated with the software system, assessing their impact and likelihood. This document is critical for understanding risk factors that may affect system safety and reliability. By identifying and analyzing hazards early in the development process, we can implement strategies to mitigate risks and ensure that the system operates safely under expected conditions.
 
  \item \textbf{Development Plan Document} \citep{Development-Plan}: The development plan contains architectural decisions and structural design rationale for the system, which are critical for validating design integrity and system stability. This document guides us in identifying key design aspects that need testing to verify adherence to design principles and effective implementation of required functionalities.
 
  \item \textbf{Module Interface Specification (MIS)} (TBD): The MIS defines the interface details for each module, including input and output data formats, parameter types, and function signatures. This document ensures that modules correctly integrate with one another, supporting our VnV plan by allowing us to focus on the correctness of inter-module communication.

  \item \textbf{User Manual (UM)} (TBD): The UM outlines the user interface and operational aspects of the system. It serves as a reference for usability testing, as it details expected user workflows and assists in validating whether the system operates as anticipated from the end-user perspective.
\end{itemize}

\section{Plan}

\subsection{Verification and Validation Team}

V\&V Team Lead: Derek
\begin{itemize}
    \item Oversees the entire verification and validation process and ensures all activities are completed according to the plan.
    \item Coordinates with other members to align V\&V activities with project deadlines.
    \item Reviews and approves test plans, reports, and results.
    \item Researches and ensures that the team has the tools needed for testing.
\end{itemize}

\noindent Developers: Emma, Damien, Temituoyo, Jad
\begin{itemize}
    \item Develop code and conduct unit testing to ensure basic functionality
    \item Address any defects or issues they identified during testing
    \item Participate in code reviews and provide feedback to ensure code implementation is of high quality
    \item Assists in setting up automated testing frameworks
\end{itemize}

\subsection*{Individual Breakdown}

\begin{enumerate}
    \item \textbf{Verification Lead - Damien}
    \begin{itemize}
        \item Responsible for overseeing all verification tasks, including:
        \begin{itemize}
            \item Ensuring all test cases are executed as planned.
            \item Validating that the test cases align with system requirements.
            \item Maintaining the traceability matrix between requirements and test cases.
            \item Reporting progress to the project manager.
        \end{itemize}
    \end{itemize}

    \item \textbf{Validation Lead - Emma}
    \begin{itemize}
        \item Responsible for ensuring the product meets user needs and client expectations:
        \begin{itemize}
            \item Conducting usability testing with representative users.
            \item Collaborating with stakeholders to confirm test results meet acceptance criteria.
            \item Managing and documenting feedback from validation testing.
            \item Ensuring functional and non-functional requirements are tested appropriately.
        \end{itemize}
    \end{itemize}

    \item \textbf{Test Engineers - Derek}
    \begin{itemize}
        \item Execute both verification and validation tasks under the guidance of the Verification and Validation Leads:
        \begin{itemize}
            \item Write and execute test cases based on requirements and design specifications.
            \item Report and document any defects found during testing.
            \item Re-run test cases after defect fixes to confirm resolution (regression testing).
            \item Support validation testing by preparing test environments.
        \end{itemize}
    \end{itemize}

    \item \textbf{Traceability Coordinator - Tuoyo}
    \begin{itemize}
        \item Focused on maintaining the alignment between requirements and test cases:
        \begin{itemize}
            \item Update the traceability matrix as new requirements or test cases are added.
            \item Collaborate with the Test Engineers to ensure all requirements are covered by test cases.
            \item Provide weekly updates on traceability coverage metrics.
        \end{itemize}
    \end{itemize}

    \item \textbf{Defect Manager - Jad}
    \begin{itemize}
        \item Manages the lifecycle of identified defects:
        \begin{itemize}
            \item Triage defects based on priority and severity.
            \item Track defect resolution and communicate updates to the team.
            \item Ensure resolved defects are verified through regression testing.
        \end{itemize}
    \end{itemize}
\end{enumerate}

\subsection{SRS Verification Plan}

\noindent Our approach for SRS verification involves understanding the following objectives:

\begin{itemize}
    \item Validate that the SRS covers the project’s functional and non-functional requirements
    \item Verify that requirements are clearly defined, consistent and testable
    \item Validate that requirements are feasible and achievable within project constraints
\end{itemize}

After understanding these objectives, we will prioritize the following steps:
\begin{enumerate}
    \item Have a team meeting and brainstorm all use case scenarios and requirements before meeting with our supervisor. We can make a SRS checklist to do this.
    \item Conduct a peer review with another team. For our capstone, another team is also doing the same project, so it will be best to work with them.
    \item Book a meeting with our supervisor and present the following use cases and requirements our platform covers.
    \item Discuss possible use cases and requirements with our supervisor. We can add these new ideas to our SRS checklist.
    
\end{enumerate}

\subsection{Design Verification Plan}

Verification Methods:

\begin{itemize}
    \item Code Reviews 
        \\ \\Team members can conduct code reviews regularly to focus on code readability, efficiency, and to ensure that our code meets design requirements and specifications.
    \item Automated Testing
        \\ \\By using an automated testing framework, we can simulate real user interactions, such as signing up for the league. These testing scripts can be run quickly and save us time.
    \item Manual Testing
        \\ \\ We can conduct testing with users involved in the league to gather feedback and ensure that the platform meets user expectations.
    \item Performance Testing
        \\ \\Check website performance under heavy load and traffic to meet user expectations. An example would be how the website runs when multiple users register teams at the same time.
    \item Security Testing
        \\ \\Check for potential security issues, such as SQL injection to prevent data from being stolen.
    \item Data Verification
        \\ \\Verify that data is being stored, retrieved, and updated accurately in our database to prevent errors and meet user expectations.
    \item Documentation Review
        \\ \\Update documentation frequently, such as user guides, system diagrams, etc. to enforce updated knowledge throughout the team.
\end{itemize}

\subsection{Verification and Validation Plan}

Checklist:

\begin{itemize}
    \item Code Reviews 
        \begin{itemize}
            \item Does our code meet readability and efficiency standards?
            \item Are there code sections we can optimize?
            \item Does the implementation match the design specifications?
        \end{itemize}
    \item Automated Testing
        \begin{itemize}
            \item Does our automated script cover the basic functionalities of our design specifications?
            \item Are there edge cases missing?
        \end{itemize}
    \item Manual Testing
        \begin{itemize}
            \item Do the manual tests reflect real scenarios?
            \item Is user feedback collected effectively and analyzed?
        \end{itemize}
    \item Performance Testing
        \begin{itemize}
            \item Does our platform handle high traffic without difficulties?
            \item Do we have performance benchmarks and are they met?
        \end{itemize}
    \item Security Testing
        \begin{itemize}
            \item Did we test against common vulnerabilities?
            \item Is sensitive data properly handled and protected?
        \end{itemize}
    \item Data Verification
        \begin{itemize}
            \item Is data transactions being done correctly?
            \item Does our system maintain data accurately?
        \end{itemize}
    \item Documentation Review
        \begin{itemize}
            \item Are our documents up to date?
            \item Are user guides easy to understand?
        \end{itemize}
\end{itemize}

\begin{table}[h]
\centering
\begin{tabular}{|l|l|l|}
\hline
\textbf{Milestone}        & \textbf{Description}                                          & \textbf{Completion Date} \\ \hline
Code Reviews              & Regular reviews focusing on readability and efficiency       & Week 2                  \\ \hline
Automated Testing         & Implement and run automated test scripts                     & Week 4                  \\ \hline
Manual Testing            & Conduct user testing and gather feedback                    & Week 6                  \\ \hline
Performance Testing       & Test website under heavy load                                & Week 8                  \\ \hline
Security Testing          & Perform security checks and vulnerability assessments        & Week 10                  \\ \hline
Data Verification         & Ensure data is accurate and properly handled                 & Week 12                  \\ \hline
Documentation Review      & Update user guides and technical documentation               & Week 14                 \\ \hline
\end{tabular}
\caption{Milestone Schedule}
\end{table}



\subsection{Implementation Verification Plan}

\begin{itemize}
    \item Code Walkthroughs
    \\ \\ Developers of a segment of code will share and present it to peers. The main focus here is design principles and coding standards.

    \item Code Inspections
    \\ \\ Peers will inspect code to find possible ways for optimization.

    \item Static Analyzers
    \\ \\ Use linting tools, such as ESLint to detect syntax errors and uphold standard code practices.
\end{itemize}

\subsection{Automated Testing and Verification Tools}

\noindent Primary Testing Tool: Playwright\\

\noindent Playwright is the most modern end-to-end testing framework. It can support testing across different browsers like Chrome, Firefox, and Safari. The support that this framework has is ideal for ensuring that our platform works smoothly across various devices and browsers.\\

\noindent Compared to other testing frameworks, Playwright offers faster execution, and built-in capabilities for handling complex UI interactions like pop-ups, a fast and growing online community, and Microsoft, a tech giant, backs it up. \\

\noindent Linting Tool: ESLint\\

\noindent ESLint will be used to catch potential issues in the code, such as syntax errors, security vulnerabilities, or deviations from coding standards, before execution.\\

\noindent Saves time in code reviews and reduces the risk of runtime errors by enforcing best practices during development.

\noindent Continuous Integration Tool: GitHub Actions\\

\noindent GitHub Actions will be used for continuous integration. At the early phases of this project, workflows will be used to automatically generate/compile documentation upon pull request merging. In the future, we may invoke workflows to deploy the web application upon merge to development branch. \\

\noindent Saves time in PDF generation and code reviews and prevents Latex errors from being pushed to main branch. \\

\subsection{Software Validation Plan}

Methods: \\

\begin{itemize}
    \item Requirement-Based Testing
    \\ \\ Create test cases based on requirements to verify that each functionality works as expected.
    
    \item Regression Testing
    \\ \\ Ensure basic functionality remains unaffected when developing new features by running a script that tests all the basic functionalities before updates are pushed.

    \item Device Testing
    \\ \\ Validate that the platform works across various devices, such as smartphones, tablets, desktops, etc.

    \item Cross-browser Testing
    \\ \\ Validate that the platform works across various browsers,
    such as Chrome, Firefox, Safari, etc.
    
\end{itemize}

\section{System Tests}

This section goes over the tests for both functional and non-functional requirements of the platform. Functional requirements tests cover core features like authentication, team management, and scheduling. These tests are focused on making sure the system does exactly what it’s supposed to, based on the requirements laid out in the SRS.

The non-functional requirements tests are split into categories to check additional qualities of the system: Look and Feel, Usability and Humanity, Performance, Operational and Environmental, Maintainability and Support, and Security. These tests make sure the platform isn’t just functional but also user-friendly, reliable, compatible across devices, and secure. Each section breaks down the tests and criteria we’ll use to evaluate these areas.

\subsection{Tests for Functional Requirements}

This section is divided into different areas to cover subsets of the functional requirements comprehensively. The following tests cover core features like authentication, team management, and scheduling. References to the SRS are provided to show the alignment with documented requirements.

It should be noted that when \texttt{Roles} are described doing actions in the functional tests (e.g. ``Captain inputs...''), these are assumed a tester acting in place of the role. 

\subsubsection{Authentication and Access Control}

This section includes tests related to the authentication of users and the enforcement of role-based access control. These tests cover requirements related to login functionality and role-specific permissions as outlined in the SRS data model. This section coveres SRS Req \#1

\paragraph{Login with Invalid Credentials}

\begin{enumerate}
\item{FR-1\\}
Control: Manual

Initial State: System at login screen, ready to accept credentials.

Input: Invalid \texttt{UserID} and a combination of valid \texttt{UserID} + Invalid \texttt{Password} for commissioner, captain, and player accounts.

Output: Unsuccessful login. The system is ready to accept new credentials.

Test Case Derivation: Invalid login credentials should result in an unsuccessful login and the user should be able to try again.

How Test Will Be Performed: Tester will attempt to log in with an invalid \texttt{UserID} and \texttt{Password}. Verify that all invalid logins are unsuccessful.

Requirements Covered: SRS Req \#1 (authentication).
\end{enumerate}

\paragraph{Login with Valid Credentials}

\begin{enumerate}

\item{FR-2\\}
Control: Manual

Initial State: System at login screen, ready to accept credentials.

Input: \texttt{UserID} and \texttt{Password} for valid commissioner, captain, and player accounts.

Output: Successful login for valid credentials.

Test Case Derivation: Valid login credentials should result in a successful login to the system. This is also how the system will differentiate users to perform actions requiring Role permissions.

How Test Will Be Performed: Tester will log in with a valid \texttt{UserID} and Password for each \texttt{Role} (Player, Captain, Commissioner) and verify access to the platform after successful login under the correct account.

Requirements Covered: SRS Req \#1 (authentication).
\end{enumerate}

\subsubsection{Team Management}

This section includes tests for the functionality of team creation and management. These tests verify that captains can create teams with unique names, manage join requests, and prevent duplication errors. This section also includes tests for the functionality allowing players to browse and request to join teams. These tests confirm that players can request to join a team and that requests are accurately reflected to captains. This section covers SRS Req \#5 and \#6.

\paragraph{Captain Registering a Team}

\begin{enumerate}

\item{FR-3\\}
Control: Manual

Initial State: User is logged in with Role set to Captain, with no existing team associated with \texttt{CaptainID}. System is on the create team page, with at least one team registered in the database.

Input: Enter a \texttt{TeamName} that already exists in the database and a Division.

Output: Unsuccessful creation of a team with a duplicate \texttt{TeamName}. Error message displayed to Captain addressing that the \texttt{TeamName} already exists.

Test Case Derivation: As a general rule, there are to be no duplicate \texttt{TeamName}. Captains should not be able to create a team with a \texttt{TeamName} that is already taken

How Test Will Be Performed: Tester (with Captain account) submits a request to create a new team using a duplicate \texttt{TeamName}. Verify that the team is not added to the database.

Requirements Covered: SRS Req \#5 (team creation).

\item{FR-4\\}
Control: Manual

Initial State: User is logged in with Role set to Captain, with no existing team associated with \texttt{CaptainID}. System is on the create team page.

Input: Enter a unique TeamName and Division.

Output: Successful team creation after submission.

Test Case Derivation: With valid input, Captains should be able to register their team into the system.

How Test Will Be Performed: Captain creates a valid new team. Verify that the team is created and added to the database. New entry should include \texttt{TeamID, TeamName, Division, CaptainID, Roster}.

Requirements Covered: SRS Req \#5 (team creation).

\item{FR-5\\}
Control: Manual

Initial State: User logged in with Role set to Captain, with an existing team associated with \texttt{CaptainID}. System is on the create team page.

Input: Enter a new TeamName and Division.

Output: Unsuccessful creation of a team. Error message displayed for multiple teams under one \texttt{CaptainID}.

Test Case Derivation: As a general rule, Captains only have one team. Captains should not be able to create a new team when they already have one registered under their \texttt{CaptainID}.

How Test Will Be Performed: Captain attempts to create a second team. Verify that the team is not entered in the database.

Requirements Covered: SRS Req \#5 (team creation).
\end{enumerate}

\paragraph{Joining a Team}

\begin{enumerate}

\item{FR-6\\}
Control: Manual

Initial State: User logged in with Role set to Player, without an existing team associated with \texttt{PlayerID}. Another user logged in as Captain of \texttt{TeamName}.

Input: Player requests to join \texttt{TeamName}.

Output: Player's request is sent to the team's Captain.

Test Case Derivation: Players should be able to send join requests to Captains for review and approval.

How Test Will Be Performed: Player requests to join a team. Verify that the associated Captain recieves the request (Refer to \textbf{FR-7}).

Requirements Covered: SRS Req \#6 (joining teams).

\item{FR-7\\}
Control: Manual

Initial State: User logged in with Role set to Captain, with an existing team associated with \texttt{CaptainID}. A player has requested to join the associated team.

Input: Captain receives and approves request.

Output: \texttt{PlayerID} is added to \texttt{Roster} of the team associated with \texttt{TeamName}.

Test Case Derivation: Players should be added to a team after a request is approved.

How Test Will Be Performed: Captain approves a pending player's join request. Verify the team updates accordingly: \texttt{Roster} updated in database and the player should have permissions to view team-specific information.

Requirements Covered: SRS Req \#6 (joining teams).

\end{enumerate}

\subsubsection{Game Scheduling and Reporting}

This section tests game scheduling and rescheduling requests, focusing on the interaction between captains and commissioners for adjusting game schedules as defined in SRS Req \#7, \#8, \#11, and \#12.

\paragraph{Automated Season Schedule}

\begin{enumerate}

\item{FR-7\\}
Control: Manual

Initial State: Teams have been registered in the system, each with \texttt{TeamIDs} and preferences for game days/times. No current season schedule exists.

Input: 
Set up a sample database file containing:
\begin{itemize}
  \item Team Availability: Preferences for game days and times.
  \item Division Assignments: Each \texttt{TeamID} is assigned a division.
  \item Game Count Requirement: Ensures each team has an equal number of games.
  \item Available Slots: The availability of the event space.
\end{itemize}

Output: The system generates an appropriate season game schedule.

Test Case Derivation: The system needs to be able to generate an initial schedule for the season based on team availability and preferences.

How Test Will Be Performed: Load the database with team availability and divisional information. Run the schedule generation process. Refer to \textbf{TPERF-2} for performance test of the generated schedule.

Requirements Covered: SRS Req \#7 (Schedule Automation).

\end{enumerate}

\paragraph{Captain Game Reporting}

\begin{enumerate}

\item{FR-8\\}
Control: Manual

Initial State: Season schedule is generated and displayed, with captain logged in. System is on game report page.

Input: Captain submits game results (\texttt{Results, ScoreTeamA, ScoreTeamB}) or forfeits for a selected game.

Output: Game report is updated and Standings are updated (\texttt{TeamID, Wins, Losses, Ties, Rank}).

Test Case Derivation: Captains should be able to report game results which should be reflected in the platform's standings/reporting.

How Test Will Be Performed: Captain reports games. Verify that the report is reflected in database and that standings are automatically updated accordingly.

Requirements Covered: SRS Req \#12 (Reporting).

\end{enumerate}

\paragraph{Rescheduling}

\begin{enumerate}

\item{FR-9\\}
Control: Manual

Initial State: Season schedule is generated and displayed, with commissioner and captain logged in. System is on scheduling page.

Input: Captain submits a reschedule request in an open slot. The request is valid and associated with the appropriate \texttt{ScheduleID, GameID, and SlotNumber}.

Output: Request is sent to Commissioner.

Test Case Derivation: Rescheduling requests from captains should be sent to the Comissioner for review and approval.

How Test Will Be Performed: Captain initiates reschedule request. Verify that the request is sent to the commissioner for approval.

Requirements Covered: SRS Req \#8 (rescheduling).

\item{FR-10\\}
Control: Manual

Initial State: Commissioner logged in. System is on schedule page with a scheduled game entry available.

Input: Commissioner applies an override (cancel or reschedule) on an existing \texttt{GameID} or approves a reschedule request.

Output: Override is reflected in the schedule. Notifications are sent to all IDs in \texttt{Roster} of teams associated with the overridden game.

Test Case Derivation: Schedule changes should be reflected in all affected user’s view and they should be notified of the changes.

How Test Will Be Performed: Commissioner applies an override on an existing game. Verify the game change (\texttt{Schedule view, SlotNumber}, and \texttt{Available Slots} are updated). Verify notifications are received by both teams.

Requirements Covered: SRS Req \#11 (game overrides).

\end{enumerate}

\subsubsection{Announcements}

This section includes tests for league-wide announcements and commissioner game overrides, ensuring that all users receive notifications of changes and announcements, per SRS Req \#9 and \#10.

\paragraph{Posting Announcements}

\begin{enumerate}

\item{FR-11\\}
Control: Manual

Initial State: Commissioner logged in. System is on announcement page.

Input: Commissioner posts a league-wide announcement to be visible across platform views.

Output: Announcement is visible to all users.

Test Case Derivation: Commissioners' posts should be viewable by all users.

How Test Will Be Performed: Commissioner posts announcement; verify visibility for all \texttt{Roles}.

Requirements Covered: SRS Req \#9, \#10 (announcements).
\end{enumerate}

...

\subsubsection{Area of Testing2}

...

\subsection{Tests for Nonfunctional Requirements}

\subsubsection{Look and Feel}
		
\begin{enumerate}

\item{TLF-1\\}

Control: Manual
					
Initial State: Web application is launched.
					
Input: Tester navigates through primary views (including login, team management, and schedule viewing)
					
Output: The platform’s interface should be modern, intuitive, and visually consistent across all views.

Test Case Derivation: Ensures a consistent, modern interface as specified.
					
How test will be performed: Manually inspect visual consistency across pages on different devices. This can be included in the results of the usability survey as well as remaining consistent with the stylings and design library used.

Requirements Covered: SRS Req 10.1.

\end{enumerate}

\subsubsection{Usability and Humanity}

\begin{enumerate}

\item{TUH-1\\}

Control: Dynamic, Manual
					
Initial State:  Web application is launched, user is on login page.
					
Input: Tester is asked to navigate through different platform views (login, announcements, standings, schedule)
					
Output: Tester successfully navigates to each view within \hyperref[MIN_NAVTIME]{\texttt{MIN\_NAVTIME}} minutes with no prior experience or external help.

Test Case Derivation: Ease of use is achieved if new users can complete essential navigation quickly and independently, as specified.
					
How test will be performed: Conduct a observational study of usability with a \hyperref[MIN_TESTERS]{\texttt{MIN\_TESTERS}} participants who have no prior experience with the platform. Record the time taken for each view. Following the tasks, participants will complete a survey (in Appendix) to provide feedback on task difficulty and ease of navigation.

Requirements Covered: SRS Req 11.1.

\item{TUH-2\\}

Control: Manual
					
Initial State:  Web application is launched, user is on login page.
					
Input: Tester views date and time fields, metric system measurements, and content language.
					
Output: Tester verifies platform displays date and time in Canadian format, uses the metric system, and all text is in Canadian English.

Test Case Derivation: Adhering to Canadian localization standards ensures that users are comfortable with the displayed information format.
					
How test will be performed: Verify date, time, and measurement units across the platform views. Ask usability testers to confirm that information formats are clear and match Canadian standards. Feedback survey included in Appendix.

Requirements Covered: SRS Req 11.2.

\item{TUH-3\\}

Control: Dynamic, Manual
					
Initial State:  Web application is launched, user is on login page.
					
Input: Tester is asked to perform certain tasks within the platform (login, create team, request reschedule)
					
Output: Tester completes each task within an average of \hyperref[AVG_TASK_TIME]{\texttt{AVG\_TASK\_TIME}} minutes with no prior experience, while easily understanding and utilizing in-app help (navigation instructions, tooltips, and help documentation) to complete tasks.

Test Case Derivation: Effective learning support is demonstrated if new users can understand and perform basic tasks using the platform's in-app guidance.

How test will be performed: Conduct a observational study of usability with \hyperref[MIN_TESTERS]{\texttt{MIN\_TESTERS}} participants who have no prior experience with the platform. Record the time taken for each task completion. Following the tasks, participants will complete a survey (in Appendix) to provide feedback on task difficulty and provided guidance.

Requirements Covered: SRS Req 11.3-11.4

\end{enumerate}

\subsubsection{Performance}

\begin{enumerate}

\item{TPERF-1\\}

Type: Automated

Initial State: Platform is set up with sample data for standings and scheduling.

Input/Condition: Tester views league standings.

Output/Result: The standings are calculated (as mentioned in FR) with 100\% accuracy.

How test will be performed: Create test cases with predefined standings. Use an automated script to perform calculations and check results against expected outcomes, ensuring all calculations are accurate.

Requirements Covered: SRS Req 12.1, 12.3.

\item{TPERF-2\\}

Type: Automated, Manual

Initial State: Platform is set up with sample data (teams and preferences) for scheduling.

Input/Condition: Tester views league schedule.

Output/Result:
\begin{itemize}
\item{The displayed schedule shows that all teams have a balanced number of scheduled games, ensuring that no team has a game count that differs by more than \hyperref[MAX_GAME_DIFF]{\texttt{MAX\_GAME\_DIFF}} games from any other team.}
\item{The displayed schedule shows that all teams play 100\% of their games within their division.}
\item{The displayed schedule is conflict-free, considering time and location.}
\item{The displayed schedule is optimized based on matching team preferences.}
\end{itemize}

How test will be performed:  The platform is prepared with a sample dataset containing multiple teams and generates a schedule, ensuring each team is assigned games within the schedule. An automated script is used to capture the output schedule and iterate through each team's scheduled games to verify the correctness of divisional match-ups, the balance of games, and log any conflicts.

Requirements Covered: SRS Req 12.4.

\item{TPERF-3\\}

Type: Dynamic, Automatic

Initial State: The web application is launched.

Input/Condition: A pair of new login credentials are added.

Output/Result: Format of credentials in database.

How test will be performed: An automated script is used to add a pair of
new credentials and check whether the information is stored in the database.

Requirements Covered: SRS Req 12.2.

\item{TPERF-4\\}

Type: Manual

Initial State: The platform is operational with a form ready for submission (e.g., team registration, game scheduling).

Input/Condition: Tester fills out the form with invalid data (e.g., missing required fields, incorrect data formats).

Output/Result:

The platform highlights the fields with errors (e.g., required fields that are empty, invalid email formats) and displays appropriate error messages  next to each highlighted field. The form submission is prevented until all errors are corrected.

How test will be performed:

Testers will intentionally leave required fields blank, enter invalid data formats  and submit the form. Testers will verify that the fields with errors are highlighted visually and informative error messages appear next to the problematic fields, indicating the nature of the errors. Testers will confirm that the information in the form is not submitted into the database until the errors are corrected and the form is resubmitted, showing a success message upon submission

Requirements Covered: SRS Req 12.4.

\end{enumerate}


\subsubsection{Operational and Environmental}

\begin{enumerate}

\item{TOPE-1\\}

Control: Manual

Initial State: Platform launched on desktop, tablet, and smartphone.

Input: Test and access all features and navigate on each device type.

Output: Platform should be fully accessible on all device types.

Test Case Derivation: Ensures multi-device compatibility.

How test will be performed: Manual testing of the platform features, responsiveness, and navigations on desktop, tablet, and smartphone.

Requirements Covered: SRS Req 13.1.

\item{TOPE-2\\}

Control: Manual

Initial State: Platform open in Chrome, Firefox, Safari, and Edge.

Input: Access all features in each browser.

Output: Platform displays correctly across major browsers.

Test Case Derivation: Confirms browser compatibility.

How test will be performed: Testers review platform views and features in each browser.

Requirements Covered: SRS Req 13.2-13.3.

\end{enumerate}

\subsubsection{Maintainability and Support}

\begin{enumerate}

\item{TMS-1\\}

Type: Manual

Initial State: The platform is launched.

Input/Condition: Tester is asked to contact support.

Output/Result: Support email receives help request from tester without any external help.

How test will be performed: \hyperref[MIN_TESTERS]{\texttt{MIN\_TESTERS}} Tester are asked to reach
support through email and record any difficulties encountered.

Requirements Covered: SRS Req 14.1, 14.2.

\end{enumerate}

\subsubsection{Security}

\begin{enumerate}

\item{TSEC-1\\}

Type: Manual

Initial State: Platform is launched, and user roles are assigned.

Input/Condition: Tester logs in as an Administrator, Team Manager, and Player and verifies correct access.

Output/Result:

\begin{itemize}
\item{The Player role has restricted access with basic features (ex. viewing schedule, joining teams)}
\item{The Captain role has access to additional team management features.}
\item{The Commissioner has access to all administrative features.}

\end{itemize} 

How test will be performed: Manually log in with a dummy user role for each level and attempt to access various features. Verify that the access levels match the defined role-based access control requirements.

Requirements Covered: SRS Req 15.1, 15.2.

\item{TSEC-2\\}

Type: Manual

Initial State: Platform is ready for data entry.

Input/Condition: Tester attempts to submit invalid information into the database (ex. team roster with missing player names, invalid team or game information).

Output/Result: The system displays helpful error messages and prevents data entries - no data should be submitted or saved for invalid inputs, preventing any unwanted database behaviour or liability.

How test will be performed: Enter invalid data into the respective forms and submit. Observe the system's responses and ensure appropriate validation messages are displayed.

Requirements Covered: SRS Req 15.1, 15.2.

\item{TSEC-3\\}

Type: Manual

Initial State: Platform login page is loaded, and the database connection is active.

Input/Condition: Tester inputs SQL injection strings into form fields such as the username and password fields.

Output/Result:

\begin{itemize}
    \item{The platform rejects input containing SQL injection patterns.} 
    \item{No unintended database behavior occurs (e.g., unauthorized login or data loss).} 
    \item{Error messages are displayed without revealing system vulnerabilities (e.g., "Invalid credentials" instead of SQL syntax errors).} 
\end{itemize}

How test will be performed: 
\begin{enumerate} 
    \item{Navigate to the login page and enter SQL injection strings in the username and password fields.} 
    \item{Attempt to log in and observe the platform's behavior.} 
    \item{Verify that no unauthorized access or system behavior is triggered.} 
    \item{Review server logs to ensure the application logs the attempt without exposing database details.} 
\end{enumerate}

\end{enumerate}

\subsubsection{Cultural}

\begin{enumerate}

\item{TCU-1\\}

Type: Manual
					
Initial State: Platform is launched in default settings.
					
Input/Condition: Tester views platform content, date, and time information.
					
Output/Result: All text is displayed in Canadian English, with time shown in the appropriate Hamilton, Ontario, Canada time zone (EST).
					
How test will be performed: Manually inspect the platform to ensure text and date/time formats match Canadian English and the local time zone.

Requirements Covered: SRS Req 16.1.

\end{enumerate}

\subsubsection{Compliance}
\begin{enumerate}
\item{TC-1\\}
Type: Functional, Manual
					
Initial State: Dummy user account exists, with personal data such as email and phone number stored in the platform.
					
Input/Condition: User initiates a request to delete their account and associated data.
					
Output/Result: All personal data associated with the account is removed from database and platform, and user receives confirmation of data deletion.
					
How test will be performed: Manually delete an account and verify that all associated data is removed. Log any unexpected time elapsed to ensure security and compliance in appropriate time.

Requirements Covered: SRS Req 17.1.

\item{TC-2\\}

Type: Manual
					
Initial State: Platform is launched.
					
Input/Condition: Tester visually inspect the platform against W3C web standards to verify adherence to industry standards (readable fonts, accessible colors, and clear navigation).
					
Output/Result: Tester verifies platform's adherence to W3C web standards.
					
How test will be performed: Manually inspect the interface against \href{https://www.w3.org/standards/}{W3C web standards} to verify font readability, contrast, and navigation.

Requirements Covered: SRS Req 17.2.
\end{enumerate}

...

\subsection{Traceability Between Test Cases and Requirements}
Note : Requirement IDs are from SRS
\begin{center}
    \begin{tabular}{ |c|c|c| }
		\hline
		\textbf{Requirement ID } & \textbf{Test Case ID(s)} & \textbf{Description of Coverage} \\ \hline
		FR-1 & \makecell{Login-Invalid, \\ Login-Valid} & \makecell{Tests login functionality \\ with valid and invalid credentials.} \\ \hline
		FR-2 & \makecell{Captain-TeamCreation, \\ Captain-TeamDuplication} & \makecell{Ensures unique team creation \\ and duplication prevention.} \\ \hline
		FR-3 & \makecell{Player-JoinTeam, \\ Captain-ApproveJoin} & \makecell{Covers team joining requests and\\ approvals.} \\ \hline
		FR-7 & Game-ScheduleGeneration & \makecell{Verifies automated schedule\\ generation.} \\ \hline
		FR-8 & Captain-GameReport & \makecell{Ensures captains can submit\\ game results.} \\ \hline
		FR-9 & Captain-RescheduleRequest & \makecell{Verifies rescheduling requests.} \\ \hline
		FR-10 & Admin-OverrideSchedule & \makecell{Validates admin ability to\\ override schedules.} \\ \hline
		NFR-11.1 & Usability-NavigationTest & \makecell{Tests user ease of navigation\\ through the platform.} \\ \hline
		NFR-12.4 & Performance-ScheduleValidation & \makecell{Verifies schedule accuracy under\\ performance constraints.} \\ \hline
		NFR-13.2 & Compliance-W3CStandards & \makecell{Confirms adherence to web\\ standards (W3C).} \\ \hline
		% \label{table:traceability_matrix}
	\end{tabular}
\end{center}
For more, refer to requirements in SRS Document () 

\section{Unit Test Description}

\wss{This section should not be filled in until after the MIS (detailed design
  document) has been completed.}

\wss{Reference your MIS (detailed design document) and explain your overall
philosophy for test case selection.}  

\wss{To save space and time, it may be an option to provide less detail in this section.  
For the unit tests you can potentially layout your testing strategy here.  That is, you 
can explain how tests will be selected for each module.  For instance, your test building 
approach could be test cases for each access program, including one test for normal behaviour 
and as many tests as needed for edge cases.  Rather than create the details of the input 
and output here, you could point to the unit testing code.  For this to work, you code 
needs to be well-documented, with meaningful names for all of the tests.}

\subsection{Unit Testing Scope}

\wss{What modules are outside of the scope.  If there are modules that are
  developed by someone else, then you would say here if you aren't planning on
  verifying them.  There may also be modules that are part of your software, but
  have a lower priority for verification than others.  If this is the case,
  explain your rationale for the ranking of module importance.}

\subsection{Tests for Functional Requirements}

\wss{Most of the verification will be through automated unit testing.  If
  appropriate specific modules can be verified by a non-testing based
  technique.  That can also be documented in this section.}

\subsubsection{Module 1}

\wss{Include a blurb here to explain why the subsections below cover the module.
  References to the MIS would be good.  You will want tests from a black box
  perspective and from a white box perspective.  Explain to the reader how the
  tests were selected.}

\begin{enumerate}

\item{test-id1\\}

Type: \wss{Functional, Dynamic, Manual, Automatic, Static etc. Most will
  be automatic}
					
Initial State: 
					
Input: 
					
Output: \wss{The expected result for the given inputs}

Test Case Derivation: \wss{Justify the expected value given in the Output field}

How test will be performed: 
					
\item{test-id2\\}

Type: \wss{Functional, Dynamic, Manual, Automatic, Static etc. Most will
  be automatic}
					
Initial State: 
					
Input: 
					
Output: \wss{The expected result for the given inputs}

Test Case Derivation: \wss{Justify the expected value given in the Output field}

How test will be performed: 

\item{...\\}
    
\end{enumerate}

\subsubsection{Module 2}

...

\subsection{Tests for Nonfunctional Requirements}

\wss{If there is a module that needs to be independently assessed for
  performance, those test cases can go here.  In some projects, planning for
  nonfunctional tests of units will not be that relevant.}

\wss{These tests may involve collecting performance data from previously
  mentioned functional tests.}

\subsubsection{Module ?}
		
\begin{enumerate}

\item{test-id1\\}

Type: \wss{Functional, Dynamic, Manual, Automatic, Static etc. Most will
  be automatic}
					
Initial State: 
					
Input/Condition: 
					
Output/Result: 
					
How test will be performed: 
					
\item{test-id2\\}

Type: Functional, Dynamic, Manual, Static etc.
					
Initial State: 
					
Input: 
					
Output: 
					
How test will be performed: 

\end{enumerate}

\subsubsection{Module ?}

...

\subsection{Traceability Between Test Cases and Modules}

\wss{Provide evidence that all of the modules have been considered.}
				
\bibliographystyle{plainnat}

\bibliography{../../refs/References}

\newpage

\section{Appendix}

This is where you can place additional information.

\subsection{Symbolic Parameters}

The definition of the test cases will call for SYMBOLIC\_CONSTANTS.
Their values are defined in this section for easy maintenance.
\begin{longtable}{|l|l|l|p{5cm}|}

\hline
parameter & value & unit & description\\
\hline
\texttt{MIN\_NAVTIME}\label{MIN_NAVTIME} & 60 & s & the minimum time for testers to navigate to a main view \\
\hline
\texttt{MIN\_TESTERS}\label{MIN_TESTERS} & 5 & n/a & the minimum testers required for a certain system test \\
\hline
\texttt{AVG\_TASK\_TIME}\label{AVG_TASK_TIME} & 3 & min & the average task time for a task completion \\
\hline
\texttt{MAX\_GAME\_DIFF}\label{MAX_GAME_DIFF} & 2 & n/a & the maximum difference of total scheduled games between two teams \\
\hline
\end{longtable}

\subsection{Usability Survey Questions}

\paragraph{Learning}
\begin{itemize}
\item{How easy was it to learn how to use the platform? (1-5)}
\item{How helpful were the instructions or help resources in learning how to use the platform? (1-5)}
\item{How confident do you feel using the platform without assistance after your initial experience? (1-5)}
\end{itemize}

\paragraph{Usability}
\begin{itemize}
\item{How intuitive do you find the navigation of the platform? (1-5)}
\item{Did you encounter any errors or issues while using the platform? If so, please describe them. (Open-ended)}
\end{itemize}


\paragraph{Accessibility}
\begin{itemize}
\item{How accessible do you find the platform in terms of visual design (e.g., color contrast, text size, font)? (1-5)}
\item{Did you encounter any errors or issues while using the platform? If so, please describe them. (Open-ended)}
\item{Do you have any suggestions for improving the accessibility of the platform?
(Open-ended)}
\end{itemize}

\newpage{}
\section*{Appendix --- Reflection}

\wss{This section is not required for CAS 741}

The information in this section will be used to evaluate the team members on the
graduate attribute of Lifelong Learning.

The purpose of reflection questions is to give you a chance to assess your own
learning and that of your group as a whole, and to find ways to improve in the
future. Reflection is an important part of the learning process.  Reflection is
also an essential component of a successful software development process.  

Reflections are most interesting and useful when they're honest, even if the
stories they tell are imperfect. You will be marked based on your depth of
thought and analysis, and not based on the content of the reflections
themselves. Thus, for full marks we encourage you to answer openly and honestly
and to avoid simply writing ``what you think the evaluator wants to hear.''

Please answer the following questions.  Some questions can be answered on the
team level, but where appropriate, each team member should write their own
response:


\begin{enumerate}
	\item What went well while writing this deliverable?
	
	There was a lot of effective communication and problem solving throughout the
	development of the VnV plan. The team was able to develop clear sections and 
	specific testing categories for features such as the authentication, access
	control, game scheduling, etc. This made is easier to organize the structure
	of the document. Using a google doc, the team was able to create an intial 
	draft with tasks assigned to team members either individually, or in sub groups.
	Following the creation of the ideas brainstormed in the draft, we created a 
	formal documentation of the VnV plan utilizing LaTeX.
	
	\item What pain points did you experience during this deliverable, and how
		did you resolve them?
	
		A challenge we faced during the formulation of this deliverable was ensuring
		the traceability between test cases and requirements listed in section 4.4. 
		The process of tracking each requirements to their respective test cases can
		easily become complex. To resolve this issue, we established a mapping method
		to simplify the traceability.
	
	\item What knowledge and skills will the team collectively need to acquire to
	successfully complete the verification and validation of your project?
	Examples of possible knowledge and skills include dynamic testing knowledge,
	static testing knowledge, specific tool usage, Valgrind etc.  You should look to
	identify at least one item for each team member.
	
		\textbf{Static and dynamic testing knowledge} will be an important skill the
		team will collectively need for the completion of verification and validation.
		Thiw will be critical for validating both functional and non functional
		requirements.
	
		\textbf{Usability and performance testing} will be a necessary skill for the 
		verification and validation, as it will be essential for the understanding
		of the human factors and performance for the non functional requirements.
	
		\textbf{Security testing} will be an important topic for the completion of
		this deliverable due to the teams emphasis on security. A proper
		understanding of cybersecurity principles will help improve the security.

		\textbf{Tool proficiency} will a skill necessary for the team to acquire to
		improve the automated testing tools, as well as the efficiency of and accuracy
		of the testing.

		\textbf{Traceability and Requirements Management} will be crucial for our team
		to understand how to link each requirement to test cases specific test cases.
		
	\item For each of the knowledge areas and skills identified in the previous
	question, what are at least two approaches to acquiring the knowledge or
	mastering the skill?  Of the identified approaches, which will each team
	member pursue, and why did they make this choice?
	
	\textbf{Static and dynamic testing knowledge - Damien} 

	Damien will work on enhancing his statis and dynamic testing knowledge with the
	assitance of online resources.
	\begin{enumerate}
		\item Taking an online course with a focus on software testing methods.
		\item Organize a peer learning session where team members individually
		search and share various testing techniques.
	\end{enumerate}
	\textbf{Usability and performance testing - Emma} 

	Emma will work on her usability and performance testing skills to improve
	the teams performance metrics.
	\begin{enumerate}
		\item Evaluate our perfomance matrix using analytical tools.
		\item Learn more about UX/UI principles using online resources.
	\end{enumerate}
	\textbf{Security testing - Tuoyo}

	Tuoyo will focus on improving his skills with the security testing to assist
	the team with improving security practices.
	\begin{enumerate}
		\item Enroll in a cybersecurity course to learn core concepts.
		\item Learn more about common vulnerablities and techniques to mitigate them
	\end{enumerate}

	\textbf{Tool proficiency - Jad} 

	Jad will work on tool proficiency to improve the teams automated processes.
	\begin{enumerate}
		\item Practice using selected tools in examples scenarios.
		\item Attend vendor webinars or tutorials for selected tools.
	\end{enumerate}

	\textbf{Traceability and Requirements Management - Derek} 

	Derek will work on the Traceability and Requirements Management to improve the 
	validation and verification of requirements.
	\begin{enumerate}
		\item Learn the use of requirement management tools to facilitate traceability.
		\item Research and study best practices in requirement management.
	\end{enumerate}
	  
\end{enumerate}


\section*{Notes}
This section will be completed at the end of Phase 2 as specified in section 21.2.2 in the SRS.

\end{document}