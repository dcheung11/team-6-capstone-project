% THIS DOCUMENT IS FOLLOWS THE VOLERE TEMPLATE BY Suzanne Robertson and James Robertson
% ONLY THE SECTION HEADINGS ARE PROVIDED
%
% Initial draft from https://github.com/Dieblich/volere
%
% Risks are removed because they are covered by the Hazard Analysis
\documentclass[12pt]{article}

\usepackage{booktabs}
\usepackage{tabularx}
\usepackage{hyperref}
\hypersetup{
    bookmarks=true,         % show bookmarks bar?
      colorlinks=true,      % false: boxed links; true: colored links
    linkcolor=red,          % color of internal links (change box color with linkbordercolor)
    citecolor=green,        % color of links to bibliography
    filecolor=magenta,      % color of file links
    urlcolor=cyan           % color of external links
}

\newcommand{\lips}{\textit{Insert your content here.}}

\input{../Comments}
%% Common Parts

\newcommand{\progname}{Software Engineering} % PUT YOUR PROGRAM NAME HERE
\newcommand{\authname}{Team 6, Pitch Perfect
\\ Damien Cheung
\\ Derek Li
\\ Temituoyo Ugborogho
\\ Emma Wigglesworth
\\ Jad Haytaoglu} % AUTHOR NAMES                  

\usepackage{hyperref}
    \hypersetup{colorlinks=true, linkcolor=blue, citecolor=blue, filecolor=blue,
                urlcolor=blue, unicode=false}
    \urlstyle{same}
                                


\begin{document}

\title{Software Requirements Specification for \progname: subtitle describing software} 
\author{\authname}
\date{\today}
	
\maketitle

~\newpage

\pagenumbering{roman}

\tableofcontents

~\newpage

\section*{Revision History}

\begin{tabularx}{\textwidth}{p{3cm}p{2cm}X}
\toprule {\textbf{Date}} & {\textbf{Version}} & {\textbf{Notes}}\\
\midrule
Date 1 & 1.0 & Notes\\
Date 2 & 1.1 & Notes\\
\bottomrule
\end{tabularx}

~\\

~\newpage
\section{Purpose of the Project}
\subsection{User Business}
\lips
\subsection{Goals of the Project}
\lips
\section{Stakeholders}
\subsection{Client}
\lips
\subsection{Customer}
\lips
\subsection{Other Stakeholders}
\lips
\subsection{Hands-On Users of the Project}
\lips
\subsection{Personas}
\lips
\subsection{Priorities Assigned to Users}
\lips
\subsection{User Participation}
\lips
\subsection{Maintenance Users and Service Technicians}
\lips

\section{Mandated Constraints}

\subsection{Solution Constraints}
Constraints that limit the design of the solution:
\begin{itemize}
    \item \textbf{Scalability}: The initial version of the platform must support 30-40 teams, with future scalability to accommodate more teams.
    \item \textbf{Device Accessibility}: The system must be optimized for desktops, tablets, and mobile phones.
\end{itemize}

\subsection{Implementation Environment of the Current System}
The current implementation environment includes:
\begin{itemize}
    \item \textbf{Operating Systems}: Compatibility with Windows, macOS, Android, and iOS.
    \item \textbf{Supported Browsers}: The platform must be fully functional on Chrome, Firefox, Safari, and Edge.
    \item \textbf{Hosting Environment}: The platform will continue to use the existing hosting environment, which has already been established for the current website. No changes to the hosting infrastructure are anticipated unless new features require additional support.
\end{itemize}

\subsection{Partner or Collaborative Applications}
There are no partner or collaborative applications required for this project.

\subsection{Off-the-Shelf Software}
No off-the-shelf software will be utilized for payment handling or calendar integration.

\subsection{Anticipated Workplace Environment}
The platform is anticipated to be used in various environments:
\begin{itemize}
    \item \textbf{Users}: The platform will be accessed by league commissioners, team captains, players, and administrators.
    \item \textbf{Device Types}: Users are expected to use the platform on personal laptops, desktops, and mobile devices.
\end{itemize}

\subsection{Schedule Constraints}
The project is constrained by a strict timeline, with key milestones that must be met:
\begin{itemize}
    \item \textbf{Development Timeline}: The platform must be completed within the designated course timeline (approximately 8 months).
    \item \textbf{Fixed Milestones}: Deliverables must be submitted according to predefined deadlines.
\end{itemize}

\subsection{Budget Constraints}
There are no budget constraints applicable to this project. The existing infrastructure and hosting environment will be reused, and no additional licensing fees are required.

\subsection{Enterprise Constraints}
\begin{itemize}
    \item \textbf{Hosting Environment}: The platform must utilize the existing hosting environment already set up for the current system, which may come with certain limitations or capabilities that need to be respected.
    \item \textbf{No External Sponsorship}: There is no specific enterprise funding or sponsoring this project, meaning there are no additional budgetary or compliance constraints from an external organization.
\end{itemize}

\section{Naming Conventions and Terminology}

\subsection{Glossary of All Terms, Including Acronyms, Used by Stakeholders involved in the Project}
The following terms are used consistently throughout this document:
\begin{itemize}
    \item \textbf{Commissioner}: Manages the entire league and oversees operations.
    \item \textbf{Captain}: Responsible for team management, including schedules and rosters.
    \item \textbf{Player}: A participant in a team who needs access to schedules and standings.
    \item \textbf{Schedule Slot}: The specific time allocated for a game.
    \item \textbf{Waiver}: A document that participants must sign before joining the league.
\end{itemize}

\section{Relevant Facts and Assumptions}

\subsection{Relevant Facts}
\begin{itemize}
    \item \textbf{Role Definition}: Specific roles such as commissioner, captain, and player exist within the system, each with unique permissions and responsibilities.
    \item \textbf{Season Timeline}: The platform will primarily be used during the softball season from April to September.
\end{itemize}

\subsection{Business Rules}
\begin{itemize}
    \item \textbf{Team Registration}: Teams must be registered by captains before the league starts.
    \item \textbf{Scheduling Requests}: Rescheduling requests must be made at least 48 hours in advance.
\end{itemize}

\subsection{Assumptions}
\begin{itemize}
    \item \textbf{Internet Access}: All users are assumed to have access to a stable internet connection.
    \item \textbf{Mobile Usage}: It is assumed that many users will access the platform using mobile devices, so a mobile-friendly interface is required.
    \item \textbf{Payment Compliance}: Players are expected to make payments via e-transfer, which will be managed externally. The platform will include an interface for the designated staff member to track which players have completed their payments, without integrating any direct payment gateway. Manually updated payment status is assumed to be correct.
\end{itemize}


\section{Relevant Facts And Assumptions}
\subsection{Relevant Facts}
\lips
\subsection{Business Rules}
\lips
\subsection{Assumptions}
\lips

\section{The Scope of the Work}
\subsection{The Current Situation}
\lips
\subsection{The Context of the Work}
\lips
\subsection{Work Partitioning}
\lips
\subsection{Specifying a Business Use Case (BUC)}
\lips

\section{Business Data Model and Data Dictionary}
\subsection{Business Data Model}
\lips
\subsection{Data Dictionary}
\lips

\section{The Scope of the Product}
\subsection{Product Boundary}
\lips
\subsection{Product Use Case Table}
\lips
\subsection{Individual Product Use Cases (PUC's)}
\lips

\section{Functional Requirements}
\subsection{Functional Requirements}
\lips

\section{Look and Feel Requirements}
\subsection{Appearance Requirements}
The platform must feature a modern, intuitive interface with consistent visual elements across all views.

\subsection{Style Requirements}
The platform shall follow a consistent color palette, typography, and layout.

\section{Usability and Humanity Requirements}
\subsection{Ease of Use Requirements}
The product must be easy to use for graduate students.

\subsection{Personalization and Internationalization Requirements}
The platform must be tailored to Canadian English, using the metric system for measurements and adhering to local date and time formats.

\subsection{Learning Requirements}
The platform must provide clear instructions and tooltips on all views to help new users quickly learn how to navigate and use key features.\\

\noindent The platform must include easily accessible help documentation for users to learn how to perform tasks and resolve issues independently.


\subsection{Understandability and Politeness Requirements}
The platform must provide clear, straightforward navigation with simply labeled menus and buttons so that minimal prior knowledge is required.

\subsection{Accessibility Requirements}
The platform shall provide basic accessibility features, such as keyboard navigation and sufficient text contrast, to ensure usability for a wide range of users.

\section{Performance Requirements}
\subsection{Speed and Latency Requirements}
The platform must respond to user actions (e.g., scheduling, score reporting) within 1 second.

\subsection{Safety-Critical Requirements}
The platform must ensure the secure and accurate storage of player personal information and waivers, preventing data loss or corruption.

\subsection{Precision or Accuracy Requirements}
The platform must calculate and display the standings with 100\% accuracy.\\

\noindent The platform must accurately match game preferences with available slots to avoid conflicts or scheduling errors and maximize adhesion with team schedule preferences.

\subsection{Robustness or Fault-Tolerance Requirements}
The platform shall handle common errors (e.g., failed database connections) without crashing, and users shall be able to retry actions if errors occur.

\subsection{Capacity Requirements}
The platform must be able to store and handle the scheduling for at minimum 50 teams in the league.

\subsection{Scalability or Extensibility Requirements}
The platform shall support up handling 50 teams without performance issues.

\subsection{Longevity Requirements}
The platform must use modern, widely supported web technologies to ensure long-term compatibility with future devices and browsers.

\section{Operational and Environmental Requirements}
\subsection{Expected Physical Environment}
\lips
\subsection{Wider Environment Requirements}
\lips
\subsection{Requirements for Interfacing with Adjacent Systems}
\lips
\subsection{Productization Requirements}
\lips
\subsection{Release Requirements}
\lips

\section{Maintainability and Support Requirements}
\subsection{Maintenance Requirements}
\lips
\subsection{Supportability Requirements}
\lips
\subsection{Adaptability Requirements}
\lips

\section{Security Requirements}
\subsection{Access Requirements}
\lips
\subsection{Integrity Requirements}
\lips
\subsection{Privacy Requirements}
\lips
\subsection{Audit Requirements}
\lips
\subsection{Immunity Requirements}
\lips

\section{Cultural Requirements}
\subsection{Cultural Requirements}
\lips

\section{Compliance Requirements}
\subsection{Legal Requirements}
\lips
\subsection{Standards Compliance Requirements}
\lips

\section{Open Issues}
\begin{itemize}
    \item \textbf{User Interface Design}: Some aspects of the user interface are yet to be fully defined. Specific visual elements and interaction flows will be determined in collaboration with stakeholders during the design phase.
    \item \textbf{User Interface Testing}: Details on the testing methodology for the user interface are still under discussion. The testing plan will include usability testing with users from the target audience to ensure ease of use.
    \item \textbf{User Payment Tracking Feature}: The implementation specifics for the payment tracking feature for staff are still being finalized. Considerations are being made regarding how the status of each player's payment will be updated and stored.
    \item \textbf{Integration of Existing Features}: There are open questions regarding the best approach to integrate and improve existing features from the current website in the new platform.
\end{itemize}

\section{Off-the-Shelf Solutions}
\subsection{Ready-Made Products}
There are no ready-made products that will be a part of our solution. The objective is to update and completely redesign the current softball league web-based platform. While features from the original platform will be transferred over to maintain continuity, the entire solution will be developed in-house based on updated requirements and best practices.

\subsection{Reusable Components}
While no third-party reusable components will be integrated, several features from the original platform, such as user authentication, team management, and scheduling, will be carried over to ensure continuity. These components will be updated and enhanced as part of the complete redesign to meet the new requirements.

\subsection{Products That Can Be Copied}
There are no external products that will be directly copied for this project. However, the redesign will reference the design and functional elements of the current website to maintain familiarity for existing users. Additionally, best practices from similar platforms will be researched and adapted to improve the user experience.


\section{New Problems}
\subsection{Effects on the Current Environment}
\lips
\subsection{Effects on the Installed Systems}
\lips
\subsection{Potential User Problems}
\lips
\subsection{Limitations in the Anticipated Implementation Environment That May
Inhibit the New Product}
\lips
\subsection{Follow-Up Problems}
\lips

\section{Tasks}
\subsection{Project Planning}
\lips
\subsection{Planning of the Development Phases}
\lips

\section{Migration to the New Product}
\subsection{Requirements for Migration to the New Product}
\lips
\subsection{Data That Has to be Modified or Translated for the New System}
\lips

\section{Costs}
\lips
\section{User Documentation and Training}
\subsection{User Documentation Requirements}
\lips
\subsection{Training Requirements}
\lips

\section{Waiting Room}
\lips

\section{Ideas for Solution}
\lips

\newpage{}
\section*{Appendix --- Reflection}

The information in this section will be used to evaluate the team members on the
graduate attribute of Lifelong Learning.  Please answer the following questions:

\begin{enumerate}
  \item What knowledge and skills will the team collectively need to acquire to
  successfully complete this capstone project?  Examples of possible knowledge
  to acquire include domain specific knowledge from the domain of your
  application, or software engineering knowledge, mechatronics knowledge or
  computer science knowledge.  Skills may be related to technology, or writing,
  or presentation, or team management, etc.  You should look to identify at
  least one item for each team member.
  \item For each of the knowledge areas and skills identified in the previous
  question, what are at least two approaches to acquiring the knowledge or
  mastering the skill?  Of the identified approaches, which will each team
  member pursue, and why did they make this choice?
\end{enumerate}

\end{document}