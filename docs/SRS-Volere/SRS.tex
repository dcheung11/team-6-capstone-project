% THIS DOCUMENT IS FOLLOWS THE VOLERE TEMPLATE BY Suzanne Robertson and James Robertson
% ONLY THE SECTION HEADINGS ARE PROVIDED
%
% Initial draft from https://github.com/Dieblich/volere
%
% Risks are removed because they are covered by the Hazard Analysis
\documentclass[12pt]{article}

\usepackage{booktabs}
\usepackage{tabularx}
\usepackage{hyperref}
\hypersetup{
    bookmarks=true,         % show bookmarks bar?
      colorlinks=true,      % false: boxed links; true: colored links
    linkcolor=red,          % color of internal links (change box color with linkbordercolor)
    citecolor=green,        % color of links to bibliography
    filecolor=magenta,      % color of file links
    urlcolor=cyan           % color of external links
}

\newcommand{\lips}{\textit{Insert your content here.}}

%% Comments

\usepackage{color}

\newif\ifcomments\commentstrue %displays comments
%\newif\ifcomments\commentsfalse %so that comments do not display

\ifcomments
\newcommand{\authornote}[3]{\textcolor{#1}{[#3 ---#2]}}
\newcommand{\todo}[1]{\textcolor{red}{[TODO: #1]}}
\else
\newcommand{\authornote}[3]{}
\newcommand{\todo}[1]{}
\fi

\newcommand{\wss}[1]{\authornote{blue}{SS}{#1}} 
\newcommand{\plt}[1]{\authornote{magenta}{TPLT}{#1}} %For explanation of the template
\newcommand{\an}[1]{\authornote{cyan}{Author}{#1}}

%% Common Parts

\newcommand{\progname}{ProgName} % PUT YOUR PROGRAM NAME HERE
\newcommand{\authname}{Team \#, Team Name
\\ Student 1 name
\\ Student 2 name
\\ Student 3 name
\\ Student 4 name} % AUTHOR NAMES                  

\usepackage{hyperref}
    \hypersetup{colorlinks=true, linkcolor=blue, citecolor=blue, filecolor=blue,
                urlcolor=blue, unicode=false}
    \urlstyle{same}
                                


\begin{document}

\title{Software Requirements Specification for \progname: subtitle describing software} 
\author{\authname}
\date{\today}
	
\maketitle

~\newpage

\pagenumbering{roman}

\tableofcontents

~\newpage

\section*{Revision History}

\begin{tabularx}{\textwidth}{p{3cm}p{2cm}X}
\toprule {\textbf{Date}} & {\textbf{Version}} & {\textbf{Notes}}\\
\midrule
Date 1 & 1.0 & Notes\\
Date 2 & 1.1 & Notes\\
\bottomrule
\end{tabularx}

~\\

~\newpage
\section{Purpose of the Project}
\subsection{User Business}
\lips
\subsection{Goals of the Project}
\lips
\section{Stakeholders}
\subsection{Client}
\lips
\subsection{Customer}
\lips
\subsection{Other Stakeholders}
\lips
\subsection{Hands-On Users of the Project}
\lips
\subsection{Personas}
\lips
\subsection{Priorities Assigned to Users}
\lips
\subsection{User Participation}
\lips
\subsection{Maintenance Users and Service Technicians}
\lips

\section{Mandated Constraints}
\subsection{Solution Constraints}
\lips
\subsection{Implementation Environment of the Current System}
\lips
\subsection{Partner or Collaborative Applications}
\lips
\subsection{Off-the-Shelf Software}
\lips
\subsection{Anticipated Workplace Environment}
\lips
\subsection{Schedule Constraints}
\lips
\subsection{Budget Constraints}
\lips
\subsection{Enterprise Constraints}
\lips

\section{Naming Conventions and Terminology}
\subsection{Glossary of All Terms, Including Acronyms, Used by Stakeholders
involved in the Project}
\lips

\section{Relevant Facts And Assumptions}
\subsection{Relevant Facts}
\lips
\subsection{Business Rules}
\lips
\subsection{Assumptions}
\lips

\section{The Scope of the Work}
\subsection{The Current Situation}
\lips
\subsection{The Context of the Work}
\lips
\subsection{Work Partitioning}
\lips
\subsection{Specifying a Business Use Case (BUC)}
\lips

\section{Business Data Model and Data Dictionary}
\subsection{Business Data Model}
\lips
\subsection{Data Dictionary}
\lips

\section{The Scope of the Product}
\subsection{Product Boundary}
\lips
\subsection{Product Use Case Table}
\lips
\subsection{Individual Product Use Cases (PUC's)}
\lips

\section{Functional Requirements}
\subsection{Functional Requirements}
\lips

\section{Look and Feel Requirements}
\subsection{Appearance Requirements}
The platform must feature a modern, intuitive interface with consistent visual elements across all views.

\subsection{Style Requirements}
The platform shall follow a consistent color palette, typography, and layout.

\section{Usability and Humanity Requirements}
\subsection{Ease of Use Requirements}
The product must be easy to use for graduate students.

\subsection{Personalization and Internationalization Requirements}
The platform must be tailored to Canadian English, using the metric system for measurements and adhering to local date and time formats.

\subsection{Learning Requirements}
The platform must provide clear instructions and tooltips on all views to help new users quickly learn how to navigate and use key features.\\

\noindent The platform must include easily accessible help documentation for users to learn how to perform tasks and resolve issues independently.


\subsection{Understandability and Politeness Requirements}
The platform must provide clear, straightforward navigation with simply labeled menus and buttons so that minimal prior knowledge is required.

\subsection{Accessibility Requirements}
The platform shall provide basic accessibility features, such as keyboard navigation and sufficient text contrast, to ensure usability for a wide range of users.

\section{Performance Requirements}
\subsection{Speed and Latency Requirements}
The platform must respond to user actions (e.g., scheduling, score reporting) within 1 second.

\subsection{Safety-Critical Requirements}
The platform must ensure the secure and accurate storage of player personal information and waivers, preventing data loss or corruption.

\subsection{Precision or Accuracy Requirements}
The platform must calculate and display the standings with 100\% accuracy.\\

\noindent The platform must accurately match game preferences with available slots to avoid conflicts or scheduling errors and maximize adhesion with team schedule preferences.

\subsection{Robustness or Fault-Tolerance Requirements}
The platform shall handle common errors (e.g., failed database connections) without crashing, and users shall be able to retry actions if errors occur.

\subsection{Capacity Requirements}
The platform must be able to store and handle the scheduling for at minimum 50 teams in the league.

\subsection{Scalability or Extensibility Requirements}
The platform shall support up handling 50 teams without performance issues.

\subsection{Longevity Requirements}
The platform must use modern, widely supported web technologies to ensure long-term compatibility with future devices and browsers.

\section{Operational and Environmental Requirements}
\subsection{Expected Physical Environment}
\lips
\subsection{Wider Environment Requirements}
\lips
\subsection{Requirements for Interfacing with Adjacent Systems}
\lips
\subsection{Productization Requirements}
\lips
\subsection{Release Requirements}
\lips

\section{Maintainability and Support Requirements}
\subsection{Maintenance Requirements}
\lips
\subsection{Supportability Requirements}
\lips
\subsection{Adaptability Requirements}
\lips

\section{Security Requirements}
\subsection{Access Requirements}
\lips
\subsection{Integrity Requirements}
\lips
\subsection{Privacy Requirements}
\lips
\subsection{Audit Requirements}
\lips
\subsection{Immunity Requirements}
\lips

\section{Cultural Requirements}
\subsection{Cultural Requirements}
\lips

\section{Compliance Requirements}
\subsection{Legal Requirements}
\lips
\subsection{Standards Compliance Requirements}
\lips

\section{Open Issues}
\lips

\section{Off-the-Shelf Solutions}
\subsection{Ready-Made Products}
\lips
\subsection{Reusable Components}
\lips
\subsection{Products That Can Be Copied}
\lips

\section{New Problems}
\subsection{Effects on the Current Environment}
\lips
\subsection{Effects on the Installed Systems}
\lips
\subsection{Potential User Problems}
\lips
\subsection{Limitations in the Anticipated Implementation Environment That May
Inhibit the New Product}
\lips
\subsection{Follow-Up Problems}
\lips

\section{Tasks}
\subsection{Project Planning}
The project will be completed over an 8-month capstone period. Following the completion of this deliverable, the detailed project plan for the upcoming phases is outlined as follows:

\begin{itemize}
    \item \textbf{Hazard Analysis 0}
        \begin{itemize}
            \item \textbf{Deadline}: October 23
            \item \textbf{Purpose}: Identify potential risks and mitigation strategies for the project.
        \end{itemize}
    \item \textbf{V\&V Plan Revision 0}
        \begin{itemize}
            \item \textbf{Deadline}: November 1
            \item \textbf{Purpose}: Outline the Verification and Validation plan for ensuring the platform meets functional requirements.
        \end{itemize}
    \item \textbf{Proof of Concept Demonstration}
        \begin{itemize}
            \item \textbf{Timeline}: November 11–22
            \item \textbf{Purpose}: Demonstrate the feasibility of key features of the platform in a proof of concept format.
        \end{itemize}
    \item \textbf{Design Document Revision 0}
        \begin{itemize}
            \item \textbf{Deadline}: January 15
            \item \textbf{Purpose}: Develop the architectural design of the platform and refine based on stakeholder feedback.
        \end{itemize}
    \item \textbf{Revision 0 Demonstration}
        \begin{itemize}
            \item \textbf{Timeline}: February 3–14
            \item \textbf{Purpose}: Present the initial version of the platform for evaluation and collect feedback.
        \end{itemize}
    \item \textbf{V\&V Report Revision 0}
        \begin{itemize}
            \item \textbf{Deadline}: March 7
            \item \textbf{Purpose}: Provide a report of the Verification and Validation process, ensuring the platform meets its objectives.
        \end{itemize}
    \item \textbf{Final Demonstration (Revision 1)}
        \begin{itemize}
            \item \textbf{Timeline}: March 24–March 30
            \item \textbf{Purpose}: Present the final version of the platform for grading and stakeholder review.
        \end{itemize}
    \item \textbf{EXPO Demonstration}
        \begin{itemize}
            \item \textbf{Timeline}: April TBD
            \item \textbf{Purpose}: Showcase the platform at the Capstone EXPO event.
        \end{itemize}
    \item \textbf{Final Documentation (Revision 1)}
        \begin{itemize}
            \item \textbf{Deadline}: April 2
            \item \textbf{Purpose}: Submit all final documentation, including user guides, source code, and project reports.
        \end{itemize}
\end{itemize}

\subsection{Planning of the Development Phases}
The development phases of the project are organized around specific deliverables and tasks, as outlined below:

\subsubsection*{Phase 1: Requirement Gathering and Initial Planning}
\begin{itemize}
    \item \textbf{Milestones Included}:
    \begin{itemize}
        \item Problem Statement, POC Plan, Development Plan (Deadline: September 23)
        \item Requirements Document Revision 0 (Deadline: October 9)
    \end{itemize}
    \item \textbf{Activities}:
    \begin{itemize}
        \item Meet with stakeholders to determine the requirements for the platform.
        \item Document functional requirements.
        \item Create the project plan.
    \end{itemize}
\end{itemize}

\subsubsection*{Phase 2: Proof of Concept and Design}
\begin{itemize}
    \item \textbf{Milestones Included}:
    \begin{itemize}
        \item Hazard Analysis 0 (Deadline: October 23)
        \item V\&V Plan Revision 0 (Deadline: November 1)
        \item Proof of Concept Demonstration (Timeline: November 11–22)
    \end{itemize}
    \item \textbf{Activities}:
    \begin{itemize}
        \item Develop and demonstrate a proof of concept for core features, including authentication and team management.
        \item Finalize the design document.
        \item Refine based on feedback.
    \end{itemize}
\end{itemize}

\subsubsection*{Phase 3: Core Development}
\begin{itemize}
    \item \textbf{Milestones Included}:
    \begin{itemize}
        \item Design Document Revision 0 (Deadline: January 15)
    \end{itemize}
    \item \textbf{Activities}:
    \begin{itemize}
        \item Implement the platform's core features, including team management, scheduling, and payment tracking functionality.
        \item Begin integration of features from the existing platform.
    \end{itemize}
\end{itemize}

\subsubsection*{Phase 4: Verification, Validation, and Revision}
\begin{itemize}
    \item \textbf{Milestones Included}:
    \begin{itemize}
        \item Revision 0 Demonstration (Timeline: February 3–14)
        \item V\&V Report Revision 0 (Deadline: March 7)
    \end{itemize}
    \item \textbf{Activities}:
    \begin{itemize}
        \item Test and validate the platform against the requirements.
        \item Present a revision (Revision 0).
        \item Make improvements based on feedback.
    \end{itemize}
\end{itemize}

\subsubsection*{Phase 5: Final Development and Demonstration}
\begin{itemize}
    \item \textbf{Milestones Included}:
    \begin{itemize}
        \item Final Demonstration (Revision 1) (Timeline: March 24–March 30)
        \item EXPO Demonstration (Timeline: April TBD)
    \end{itemize}
    \item \textbf{Activities}:
    \begin{itemize}
        \item Make final revisions.
        \item Prepare for the final demonstration.
        \item Present the completed platform to stakeholders and at the Capstone EXPO event.
    \end{itemize}
\end{itemize}

\subsubsection*{Phase 6: Documentation and Wrap-Up}
\begin{itemize}
    \item \textbf{Milestones Included}:
    \begin{itemize}
        \item Final Documentation (Revision 1) (Deadline: April 2)
    \end{itemize}
    \item \textbf{Activities}:
    \begin{itemize}
        \item Complete all project documentation, including the final revision of the requirements, user guide, and other deliverables.
        \item Submit the final documentation for grading.
    \end{itemize}
\end{itemize}

\section{Migration to the New Product}
\subsection{Requirements for Migration to the New Product}
To migrate from the existing platform to our redesign involves several key steps to ensure that the transfer of all relevant data, features, and functionalities is seamless and complete:

\begin{itemize}
    \item \textbf{Data Preservation}: All current data from the existing platform must be preserved and migrated to the new platform. The specific data to be transferred will be confirmed through consultation with the current platform administrator, ensuring that only the necessary and relevant data is included in the migration.
    \item \textbf{Feature Continuity}: Ensure that all features present in the original platform are retained in the redesigned version, with enhancements and improvements where applicable.
    \item \textbf{Minimizing Downtime}: The softball season begins in April, which falls around when the project is expected to be completed. Due to this tight deadline, the migration process should be planned to minimize downtime for users, ensuring that the transition is as smooth as possible with minimal impact on league operations.
\end{itemize}

\subsection{Data That Has to be Modified or Translated for the New System}
During the migration process, specific data elements may need to be modified or translated to align with the new platform's structure and data requirements:

\begin{itemize}
    \item \textbf{User Accounts and Roles}: The login system needs a complete redesign. User account information, including usernames, passwords, and roles (commissioner, captain, player), will need to be reformatted to match the requirements of the new authentication system.
    \item \textbf{Team and Match Data}: Team rosters, match schedules, and historical results may require adjustments to fit the new database schema. This may involve reformatting data fields or translating old data structures to new formats.
\end{itemize}

\section{Costs}
The capstone supervisor expressed minimizing costs. Our approach will be done with free frameworks, such as React.js, and affordable hosting plans such as AWS or Google Cloud. Our team will emphasize core functionality, and avoid creating unnecessary features and costs that could undermine the product goal. By following these cost-conscious strategies, the project can stay within budget while delivering a functional and scalable platform.

\section{User Documentation and Training}
\subsection{User Documentation Requirements}
\begin{itemize}
    \item \textbf{Comprehensive User Guide}: Create a detailed user manual page tab that outlines:
    \begin{itemize}
        \item \textbf{Getting Started}: Instructions on how to register, create teams, and navigate the platform before delving deep into more advanced features.
        \item \textbf{Feature Descriptions}: Written and visual explanations of each feature, such as team management, scheduling, score tracking, etc.
        \item \textbf{FAQs}: Address common questions and troubleshooting tips to help users resolve issues independently. If unresolved, prompt users to email for more support or direct them to the discussions page.
    \end{itemize}
    \item \textbf{Interactive Tutorials}: Create pop-up icons that demonstrate key functionalities, making it easier for both new and existing users to understand how to use the platform effectively.
\end{itemize}

\subsection{Training Requirements}
\begin{itemize}
    \item \textbf{Help Support Desk}: A dedicated support email or live chat feature where users can ask questions or report issues with someone real.
    \item \textbf{Feedback Section}: Implement a section where users can provide feedback through open-ended surveys to help improve the platform.
    \item \textbf{Discussion}: A section where users can publicly post their questions and other users are free to answer them.
\end{itemize}

\section{Waiting Room}
For the softball league platform, the “waiting room” will most likely be the feature for players who have registered, but do not have a pre-formed team. These players are called “free agents,” who may be added to teams later once all team rosters are confirmed. This concept is similar to how IMLeague is handled, which is the McMaster Intramural League platform. For example, a student desires to play softball and registers for the platform. As pre-formed teams are registered and confirmed, this student will be assigned to one of the teams that have extra space on their roster. Hence, they are “waiting” for a team to join.

\section{Ideas for Solution}
Players who sign up without a pre-formed team will be placed in a waiting room, designating them as “free agents.” This waiting room is a holding area until all team rosters are confirmed. Other teams can choose from this holding area if they need any players before the final roster confirmation date. However, all players will eventually be auto-assigned to a team if the waiting room is still non-empty. Players from the waiting room will be notified of status updates. Players who register late will also be automatically assigned from the waiting room.

Therefore, we need a database to hold the status of each player in the waiting room. On the front user interface, the waiting room should be displayed to all users, so teams can see available free agents. For users with team manager permissions, they will be able to invite free agents from the waiting room to their teams if desired.


\newpage{}
\section*{Appendix --- Reflection}

The information in this section will be used to evaluate the team members on the
graduate attribute of Lifelong Learning.  Please answer the following questions:

\begin{enumerate}
  \item What knowledge and skills will the team collectively need to acquire to
  successfully complete this capstone project?  Examples of possible knowledge
  to acquire include domain specific knowledge from the domain of your
  application, or software engineering knowledge, mechatronics knowledge or
  computer science knowledge.  Skills may be related to technology, or writing,
  or presentation, or team management, etc.  You should look to identify at
  least one item for each team member.
  \item For each of the knowledge areas and skills identified in the previous
  question, what are at least two approaches to acquiring the knowledge or
  mastering the skill?  Of the identified approaches, which will each team
  member pursue, and why did they make this choice?
\end{enumerate}

\end{document}